\documentclass{../../common/thesisbyxetex}

\usepackage{enumitem}
\setlist{nolistsep}	% отступы между элементами перечесления

\addbibresource{../../common/bibliography.bib}

\begin{document}

\hypersetup{
pdftitle = {Психологические факторы формирования отцовства},
pdfauthor = {Станкевич Наталия Александровна},
pdfsubject = {курсовая работа},
pdfkeywords = {отцовство, стили, курсова}
}% End of hypersetup
 
 Психологические факторы формирования отцовства
Станкевич Наталия Александровна,
психология, 1 курс, второе высшее



\section*{Введение} 

Трансформации, затронувшие институт семьи в последней трети ХХ - начале ХХI вв.,  спровоцировали
широкую научную дискуссию, в фокусе которой оказались такие проблемы как форма семейных
отношений \cite{gay, legfat}, фактическое содержание мужских и женских ролей \cite{mercoh, percep},
новые тенденции в детско-родительских отношения и отношениях в паре \cite{relot, sex}.

Одной из наиболее значимых тем в этой дискуссии становится тема родительства. Традиционно при ее
обсуждении особое внимание уделялось матери: ее отношениям с детьми разного возраста и разного
пола \cite{maler}, возможности совмещения материнства и профессиональной самореализации,
социокультурным особенностям реализации материнской роли и т.д. В последнее время, однако,
размывание гендерных ролей и серьезные изменения в организации и функционировании семьи
вывели на первый план вопросы, связанные с ролью мужчины в этих процессах. Отцовство, будучи одной
из ключевых сфер самоидентификации мужчины \cite{imaf}, приобретает в этой связи особую
актуальность.

Глобальное смешение гендерных ролей является  важным явлением, оказывающим
непосредственное влияние на качество осуществления отцовских практик. Хотя тенденции, связанные
с этим явлением, достаточно очевидны (например, женщины уделяют гораздо большее внимаиние
профессиональной реализации, чем это было раньше), не существует комплексных исследований
влияния подобной ситуации на самоопределение мужчин в качестве отцов: <<Отцы играют множество
ролей в современных семьях, и каждая из них предполагает определенный набор идей, умений и
поведенческих реакций. Существует информация о проведении исследований некоторых наиболее
значительных составляющих тех или иных отцовстких ролей (например, эмпатии), однако
практически ничего не известно о том, каким образом и в связи с какими обстоятельствами эти
составляющие объединяются в единое целое и взаимодействуют в реальной жизни>>
 \cite[131]{f21}.

%Исследования феномена отцовства  ведутся  в различных плоскостях: в психологическом, социальном,
%культурном измерении, о представлениях об отцовстве мужчин и ожиданиях
%и представлениях женщин о <<хорошем отце>>.

Российский исследователь И.С. Кон указывал, что отцовство может быть исследовано как <<социальный
институт, то, как его представляет себе общество, [и] отцовство как деятельность, практики и
стили поведения. Для исследования этих явлений нужны разные источники и методы. В первом случае
осуществляется реконструкция и анализ социокультурных норм, чего общество ожидает от отца "вообще".
Во втором происходит описание и анализ того, что фактически делают и чувствуют конкретные отцы,
какова психология отцовства \cite[3]{konot}.

Созвучна этому и мысль Ю.В. Борисенко, который выделяет два подхода к изучению родительства и, в
частности, отцовства: <<Существует два подхода к изучению родительства, в зависимости от того, кто
считается отправной точкой изучения - ребенок или родитель. Первый, наиболее распространенный,
подход рассматривает родительство применительно к развитию ребенка [...], во втором подходе
рассматривается выполнение родительской роли через призму личности родителя. Здесь исследуется
самореализация личности в родительстве, вводятся понятия «социальная роль», «статус», «социальные
нормы», «стереотипы и требования», исследуется феномен так называемого родительского «инстинкта»
(материнского и отцовского), исследуются чувства, образы - Я, Я-концепция и другие личностные
характеристики, так или иначе связанные и изменяющиеся с родительством \cite[11]{psyot}

В соответствии с выбранной темой курсовой работы, интерес для нас будет представлять массив
литературы --- как теоретической, так и описывающей конкретные исследования, --- относящийся ко
второму направлению в этой классификации, а именно к психологическому измерению отцовства.
Социокультурные же нормы, ожидания общества мужчин в роли отцов, стереотипы и т.д. будут
рассматриваться нами лишь в той степени, в которой они влияют на психологию отцов и конкретные
методы реализации ими отцовских функций.

Актуальность темы исследования определяется, во-первых, высокой социальной значимостью проблем,
касающихся функционирования и развития семьи на современном этапе. Многочисленные и разнообразные
факторы, оказывающие как явное, так и скрытое влияние на формирование представлений о семье,
детско-родительских и партнерских отношениях, изменения, происходящие в конфигурации семьи, системе
семейных отношений и в законах развития семьи как системы, делают исследования в этой области
весьма востребованными.

Во-вторых, несмотря на наличие огромного объема теоретической литературы и
практических исследований, посвященных материнству и детско-материнским отношениям, тема отцовства
еще не получила достаточного освещения. В существующих публикациях, кроме того, тема отцовства
рассматривается с точки зрения взаимодействия отца и ребенка, и того влияния, которое наличие
или отсутствие отца, а также то, каким именно образом он взаимодействовал с ребенком, повлияло на
последнего. Лишь немногие работы рассматривают отцовство с точки зрения мужчины,
понимая его как сложный социокультурный, психологический и экзистенциальный опыт конкретных людей.
Еще меньше исследований посвящено вопросам формирования отношения мужчин к отцовству:
тому, что влияет на становление отцом, на формирование образа отцовства в целом и собственного
отцовства в частности. Не систематизированы факторы, оказывающие влияние на то, каким отцом станет
данный мужчина, какие отцовсткие практики будет использовать во взаимоотношениях со своими детьми:
<<Ни один исследователь еще не описал пути, которые ведут мальчика к осуществлению отцовских
функций; не была даже разработана единая теория, учитывающая весь комплекс процессов развития,
которые придают смысл и очертания тем или иным отцовским практикам. Кроме того, относительно мало
известно о факторах, которые изменяют степень вовлеченности отцов в жизнь их детей с течением
времени>>  \cite[131]{f21}.

В связи с этим целью данной работы является анализ психологических факторов
формирования отцовства.

Достижение поставленной цели предполагает решение следующих задач:

\begin{enumerate}
	\item Дать определение понятию “отцовство”, раскрыть структуру отцовства как психологического
феномена;
	\item Определить современную социокультурную специфику отцовства;
	\item Выявить основные психологические факторы формирования отцовства.
\end{enumerate}

Объектом данного исследования является феномен отцовства в психологии.

Предмет исследования – психологические факторы формирования отцовства.


\chapter{Отцовство в современном социо-культурном контексте}

\section{Тенденции}

Одним из важнейших изменений, оказывающим существенное влияние на представления об отцовстве,
являются
изменения в структуре и функциях семьи,  включающие в себя следующие процессы:
во-первых, меняется конфигурация семьи: на смену нуклеарной семье приходит расширенные
ее модели --- семьи с одним родителем, бездетные семьи, семьи с однополыми родителями и т.д.
Понимание отцовства
в этой связи также претерпевает изменения, отходя от классической модели, в которой процедура
установлени отцовства основывалась на предположении о том. что <<отцом ребенка является, скорее
всего
 тот, кто имел сексуальные отношения с его матерью, а этим человеком, вероятнее всего,
является муж>> \cite[318]{legfat}

Во-вторых, происходит переход от понимания семьи как структурной единицы --- <<ячейки общества>>,
--- к семье как процессу. На первое место выдвигаются не элементы семьи, субсистемы семейной
системы, --- а отношения между ее членами, которые и создают то, что называется <<быть семьей>>
\cite{fam}. В этой связи состав  --- возраст, пол и количество  членов какой-либо группы уже не так
существенны для того, чтобы определить  ее как семью. \cite[15]{fatpsy}.

Будучи погружено в определенный социо-культурный и исторический контекст, который определяет
видение обществом
отцовской роли, оно неотделимо от него.

В современной ситуации отцовства можно выделить две значительные, противоположные
тенденции. С одной стороны  в отцовстве наблюдаются разрушительные явления: растет количество
разводов и, соответсвенно, неполных семей, дети в которых растут, не поддерживая значимых отношений
с отцами. Согласно проведенным исследованиям, лишь небольшой процент отцов, дети которых
родились вне брака, продолжали жить совместно с ними и их матерями либо поддерживать с ними близкие
отношения спустя пять лет \cite{long}. В результате появляется все больше мужчин, не имеющих
отцовского образа, не имевших возможности сформировать для себя позитивную модель мужественности и
получить пример детско-отцовских отношений. Эта ситуация весьма актуальна на современном
постсоветском пространстве.

Все это позволяет исследователям говорить о кризисе традиционной семьи и, в частности, отцовства.
Необходимо, однако, отметить, что в синергетическом понимании кризис представляет собой некую точку
бифуркации, преодоление которой предполагает либо разрушение системы, либо вывод ее на качественно
иной уровень функционирования, предполагающий новые условия и законы развития и новое состояние
системы в целом.

Касательно этих тенденций исследования отцовства могут быть сгруппированы на основании той оценки,
которую их авторы им дают: если одни воспринимают их как кризис семьи, а <<вовлеченное отцовство>>
как вызов, то другие, как, например, И.С. Кон, предполагают, что изменение роли мужчин в семейных
отношениях --- это часть естественного процесса размывания гендерных ролей и ослабления
традиционного противопоставления между ними \cite{konmen}

\section{Основные понятия}

Одной из проблем, обсуждающихся в литературе, посвященной отцовству, яяляется проблема определения
этого понятия. Существует несколько подходов в определении отцовства: оно может рассмотрено как
определенные отношения, существующие между отцом и ребенком; как личный опыт и сфера
самоидентификации мужчины; в определении отцовства могут быть отражены структурные элементы
отцовства, а также функции, которые отец выполняет по отношению к ребенку и т.д.

Данная работа направлена на изучение отцовства с точки зрения мужчины, как важнейший элемент его
экзистенциального опыта. В этот опыт входят предтсавления мужчины о себе как об отце.
его отношение к отцовству вообще и собственному отцовству в частности. его
эмоциональные реакции на различных этапах отцовства, его знания о детско-отцовских
отношениях, убеждения, система норм и ценностей, сввязанная с воспитанием детей,
представления о своей роли в этом процессе, о своих функциях и т.д.

Ю.В. Борисенко предлагает формулировать определение отцовства, не основываясь на таких понятиях.
как чувства и инстинкты, в связи с чем его определение отцовства звучит следующим образом:
<<отцовство можно определить, как категорию психологии
личности, отражающую основные этапы развития личности, характеризующую
комплекс интегральных, социальных и индивидуальных характеристик личности,
проявляющихся на всех уровнях жизнедеятельности человека: эмотивно-
аксеологическом, когнитивном и операциональном; включающую в себя оценочный
компонент и необходимость выполнения следующих функций: защитной - как
кормилец и защитник; презентативной — как персонификация власти, воспитатель
и высший дисциплинизатор; ментальной - как пример для подражания;
социализирующей — как наставник во внесемейной общественной деятельности и
отношениях, транслятор социальных норм, фигура, обеспечивающая связь
поколений>> \cite[48]{psyot}.

Представляется. однако, что эмоциональный компонент, иными словами, чувства мужчины, связанные с
отцовством, включающие в себя отношение к себе как к отцу и отношение к ребенку, является одним из
ключевых в сруктуре отцовства, и потому не может быть исключен из его определения. При его
отсутствии данное определение может обрести слишком широкий характер и утратить соответствие
определяемому понятию. Так, у Ю.В. Борисенко выделены структурные компоненты отцовства и роли,
которые отец играет (может играть) по отношению к ребенку. При этом сам феномен отцовства
определяется как <<категория психологии
личности>>, что явно является слишком широким определением данного понятия. В то же время, не
представляется необходимым выделять в определении отцовства отцовсткие роли. поскольку их набор
весьма вариативен, а динамика изменений этих ролей представляет собой широкую тему для отдельного
исследования.

Более  близким к реальным способам реализации отцовских практик представляется определение. которое
А.Я. Варга дает родительскому отношению: <<родительское отношение --- это целостная система
разнообразных чувств по отношению к ребенку, поведенческих стереотипов, практикуемых в общении с
ребенком, особенностей восприятия и понимания характера ребенка, его поступков>> \cite[7]{varga}.

Несмотря на то, что понятие родительства изначально шире, чем понятие отцовства, данное
определение, на наш взгляд, раскрывает сущность отцовства как системы взаимоотношений отца и
ребенка, проявляющихся на эмоциональном, когнитивном и деятельностном уровнях.  В этом определении
отсутвуют компоненты структуры отцовства, однако они могут варьироваться и потому их перечисление
может быть излишним в базовом определении понятия.

Представляется, что отцовство может быть определено как комплексный экзистенциальный опыт мужчины,
связанный с построением отношений с ребенком и проявляющийся на эмоциональном.
когнитивном и деятельностном уровне.

Данное определение включает в себя следующие базовые представления об отцовстве:

\begin{itemize}
	\item Отцовство представляет собой экзистенциальный, т.е. жизненно важный опыт
мужчины;
\item Этот опыт неразрывно связан с отношениями мужчины с его ребенком \ детьми (в данном случае не
столь важно разделение на биологических и приемных детей -- базовым является отношение к ним
мужчины как к <<своим>> детям);
\item Отцовство, далее, это комплексный, т.е. сложный феномен, обладающий определенной структурой;
\item Отцовство проявляется в различных сферах: эмоциональная сфера -- это комплекс чувств, который
мужчина испытывает к ребенку, а так же к себе как к отцу; когнитивная -- это сововкупность
представлений мужчины об отцовстве, его суждения о тех или иных моделях воспитания, его знания об
особенностях реализации отцовских практик, а также -- его представления и фантазии о себе как отце
еще до рождения ребенка; деятельностной -- конкретные действия и поведенческие схемы. реализуемые
отцов в ходе общения со своим ребенком.

\end{itemize}

Представляется, что понятие сферы более уместно в данном случае, чем понятия уровней -- последнее
предполагает некую иерархию, в то время как в реальности и чувства, и знания, и действия
присутсвуют одновременно, не обладая при этом большей или меньшей значимостью по сравнению друг с
другом.

Другой важнейшей категорией, рассматриваемой в работе, является формирование отцовства. Это сложный
процесс, включающий в себя различные этапы, в ходе которого формируются
представления мужчины о самом себе (как личности, мужчине,
профессионале и тп.), о себе в качестве отца, а так же его представления об идеальном отце ---
каким он должен быть, каковы наиболее
важные его качества, а так же о конкретной реализации отцовской роли по отношению к своим детям.
Сюда же могут
включаться оценка своих действий как отца (в случае, когда если уже есть дети), а так же
предположения о том, каким отцом мужчина сможет стать, анализ различных
сложностей в реализации отцовских функций --- настоящих или предполагаемых, а также о положительных
сторонах отцовства.

Таким образом, под формированием отцовства мы будем понимать  формирование представлений об
идеальном отце и формирования образа себя как отца.






\section{Структура отцовства}

Как было отмечено выше, отцовство представляет собой сложный  социокультурный и психологический
феномен, характеризующийся определенной стркутурой.

А.Я. Варга, детально изучавшая феномен родительства, выделила в нем следующие структурные
компоненты:

<<Когнитивная составляющая родительского отношения содержит знания и
представления о различных способах и формах взаимодействия с ребенком, об
оптимальной близости в отношениях; знания и представления о целевом аспекте этих
взаимоотношений, а также убеждения в приоритетности тех направлений
взаимодействия с ребенком, которые реализуют родители.

Поведенческая составляющая представляет собой формы и способы поддержания
контакта с ребенком, формы контроля, воспитание взаимоотношениями, путем
определения дистанции общения.

Эмоциональная составляющая включает оценки и суждения о различных типах
родительского отношения, а также доминирующий эмоциональный фон,
сопровождающий поведенческие проявления родительского отношения>> \cite{varga}

Иными словами, в структруре родительского отношения к детям можно выделить знания и опыт родителей,
их эмоциональное отношение к ребенку (детям) и тем или иным методам их воспитания, а так же те
способы построения взаимоотношений с детьми,
которые они применяют, исходя из предыдущих двух факторов.

Представляется, что эти составляющие,
наряду с перечисленными выше, могут применяться не только к родительским отношениям в целом, но
так же и к детско-отцовскому взаимодействию.

Данная структура соответствует пониманию отцовства как индивидуального психологического опыта
мужчины, помещая во главу угла его мысли, эмоциональные состояния и поведенческие реакции.
Несомненно, что данные компоненты действуют в тесной взаимосвязи, и на практике далеко не всегда
отделимы друг от друга.

Содержание перечисленных компонентов родительского (отцовского)
отношения может быть расширено. Так, под когнитивной составляющей, помимо вышеперечисленного, можно
понимать не только собственные знания и представления мужчины о его отцовстве, но и его
знания его семейной истории, способность видеть и адекватно оценивать тенденции детского-отцовских
отношений, прослеживающиеся в разных ее поколениях. Кроме того. к когнитивной составляющей могут
быть отнесен знания мужчины о социо-культурных особенностях восприятия отцовства в его окружении,
об образах <<хорошего>> и  <<плохого>> отца, закрепленных в коллективном
бессознательном.

В поведенческом компоненте
представляется необходимым акцентировать его двустороннюю направленность. Реализация тех или иных
отцовских практик зависит не только отличности отца, но и от ребенка: его пола, возраста, количества
детей в семье, порядка их рождения, личностных психологических и физических особенностей и т.д.
Это те конкретные действия,которые отец применяет (или находит желательныи
применять) по отношению к ребенку. Очевидно также, что реализация тех или иных отцовских практик
вплетена в
сложную систему взаимоотношений в конкретной семье, а результаты ее связаны с реакциями на
нее ребенка и других членов семьи, от типа коммуникации в семейной системе, ее структуры и т.д.

Эмоциональная составляющая отцовского отношения, на наш взгяд, помимо вышеперечисленного может
включать личностные особенности конкретного мужчины: его самовосприятие вообще и в качестве отца;
уровень его психической организации, степень готовности и желания исполнять отцовские функции;
его отношения с его собственным отцом, восприятие им отношений между его родителями, его
эмоциональные переживания и впечатления, связанные с его взаимодействием с отцом, его желания,
страхи, потребности, реакции на социокультурные требования, предъявляемые отцам в том или ином
обществе и т.п.

Будучи рассмотрено в качестве внутреннего экзистенциального опыта
конкретного мужчины безотносительно к внешним его проявлениям, отцовство может включать в себя:

\begin{itemize}
	\item  совокупность свойств сознания человека, таких
как: потребности, влечения, желания, установки, ценностные ориентации,
мировоззрение, «образ Я», «Я-концепция»;

\item принятие социальных требований к отцу, принятие социальной роли отца, принятие статуса отца в
социуме, принятия своих отцовских чувств;

\item проблема развития самосознания мужчины, проблема самооценки и самоконтроля, развития осознания
себя отцом, необходимости соблюдения социальных норм, собственной значимости для ребенка
\cite{psyot}
\end{itemize}



\section{Роли отца в структуре семьи.}

Исследования отцовства у многих авторов, как русскоязычных \cite{relot, psyot}, так и зарубежных
\cite{meta, morfat, legfat}, начинаются с констатации факта глубинных изменений, которые происходят
в понимании и реализации мужчинами своей роли в качестве отцов. Если традиционное понимание
отцовской роли сводилось к отцу-<<добытчику>>, <<примеру для подражания>>, а также
<<персонификации власти>> в отношении детей, --- то новое видение подразумевает гораздо большую
вовлеченность отца в процесс воспитания. В эмоциональном измерении это выражается в  эмоциональной
ангажированности отца в процессы воспитания и взаимодействия с детьми, его близости, открытости и
доступности для ребенка. Процессуальное измерение нового типа отцовства включат такие параметры как
внимание, которое отец уделяет ребенку, их общие занятия, время, проведенное вместе, эмоциональный
фон их отношений и т.д.
Именно эмоциональная вовлеченность в отношения между людьми является основой нового формирующегося
представления о семье как процессе, а не структурной единице, ячейке общества. Очевидно, что такое
понимание семьи открывает новые перспективы в видении детско-родительских, в том числе
отцовско-детских отношений, акцент в них также делается на отношениях, на поддержании
аффективной
связи между отцом и ребенком \cite[15]{fatpsy}.

Соответственно, меняются роли, которые, как предполагается, отец должен выполнять в семье; меняется
и сам образ <<хорошего>> и  <<плохого>> отца. На смену отцу - главе семьи, <<персонификации
власти>>,
<<нравственному авторитету>> пришел отец - <<кормилец>>, <<добытчик>> и т.д. Соответственно,
отцовский авторитет, ранее основавшийся на моральных и традиционных нормах, начал определяться
критериями профессиональной успешности мужчины, его карьерного роста, дающего ему возможность
содержать жену и детей. При этом, как в случае случае отца-<<главы>>, так и в случае отца -
<<кормильца>> не предполагалась активное включение отца в процессы воспитания детей и повседневного
ухода за ними --- как эти, так и другие бытовые обязанности возлагались на мать -- <<хранительницу
домашнего очага>>. Таким образом, существовали четко фиксированные семейные роли мужчины и женщины
в качестве отца и матери; сферы их активностии ответственности практически не пересекались.

Современная ситуация, складывающаяся вокруг отцовства, позволяет говорить о размывании границ между
этими ролями. Активная включенность отцов в общение и уход за детьми, теплая, аффективная связь
между ними становится позитивной и желательной моделью поведения, которая в некоторых странах
получает и законодательное закрепление: так, в Швеции существует специальный отпуск для мужчин в
связи с рождением ребенка. Другими словами, на современном  этапе от мужчины требуется не только
обеспечивать семью
материально, но и уделять
достаточно времени, внимания и заботы детям.

В практическом плане эти требования отражаются в дилемме соотношения работы (а вместе с тем
профессиональной успешности, самореализации и т.д.) и семьи для мужчины.
В этом плане выбор мужчиной количества часов работы,
продолжительности рабочего дня, места работы, значимости карьерного роста становится не только
вопросом успешности и финансовой независимости,
но и приобретает моральное звучание. София Бъорк в своем исследовании, посвященном
продолжительности рабочего дня шведских отцов подчеркивает, что значимым аспектом  выбора
между частичной или полной занятостью становится необходимость <<показать себя хорошим и
ответственным отцом в глазах окружающих>> \cite[221]{morfat}. <<Хороший отец>>, таким образом, это
уже не просто отец, который, сделав успешную карьеру, может достойно обеспечивать жену и детей ---
это человек, который сохраняет и развивает эмоциональный контакт с детьми, дает им заботу и тепло,
участвует в повседневном уходе за ними --- один словом, берет на себя функции и семейные роли,
традиционно считавшиеся женскими.

В этой связи необходимо упомянуть такие понятия как <<вовлеченное отцовство>>,
при котором мужчина принимает активное участие в уходе за детьми и их воспитании, а так же
новый,  только формирующийся и пока не получивший широкого распространения  тип <<нового
отцовства>> (<<new fatherhood>>), когда мужчина полностью разделяет домашние обязанности и хлопоты
по уходу за детьми со своей женой, или даже полностью берет на себя эту роль, оставляя работу, в то
время как фунцию материального и финансового обеспечения семьи берет на себя женщина.

Динамику трансформации представлений о роли отца в семье можно проследить на примере изменений
предпочтений в выборе тех или иных отцовских практик -- методов воспитания и поведенческих
стереотипов, которые отец применяет или считает нужным применять по отношению к ребенку.
Совокупность отцовских практик может быть охарактеризована понятием <<стиль отцовства>>.

Термин <<стиль отцовства>> используется как в русскоязычной, так и зарубежной литературе. В
английском языке существуют  отдельные термины для отцовства как
состояния fatherhood и  отцовства как деятельности  --- fathering. Таким образом, стиль отцовства
может быть определен как совокупность методов реализации мужчиной
его отцовской роли, его приоритетов в воспитании детей, поведенческих реакций, эмоционального фона
и особенностей его взаимоотношений с детьми.
Понятие <<стиль отцовства>> тесно связано с понятиями стиля родительства, стиля
воспитания, отцовских практик  и т.д.


Стили воспитания различаются по таким критериям как эмоциональная окраска общения родителей с
ребенком; мера и способ осуществления контроля; распределение семейных ролей между детьми и
взрослыми; степень вовлеченности каждого из родителей в воспитание, общение и ежедневный уход за
ребенком и т.д. \cite{stil}. Те же критерии можно применить и для детско-отцовских отношений.

Дэн Аллендер в своей книге <<Как дети воспитывают родителей>> выделил в структуре
детско-родительских (детско-отцовских) отношений эмоциональный и функциональный аспекты: суть
детско-родительских
интеракций, согласно Аллендеру, состоит из двух вопросов, которые дети вербально или
через свои действия задают каждому из родителей: <<Любишь ли ты меня?>> (эмоциональный аспект) и
<<Могу ли я делать все что угодно?>> (функциональный аспект) --- и ответов, которые родители
дают на эти вопросы своими словами или действиями \cite{den}. Сочетние этих ответов формирует тот
или иной стиль воспитания.

Существуют различные типологии стилей родительства. Можно выделить такие их классификации как:

\begin{enumerate}
	\item Демократический и авторитарный стили воспитания (Д.М. Болдуин);
	\item Авторитетный, авторитарный, снисходительный и безразличный (Д.Баумринд, Э.Маккоби, Дж.
Мартин);
	\item Автократичный, авторитарный, демократичный, эгалитарный, разрешающий, попустительский,
игнорирующий (Д. Элдер) \cite{stil};
	\item Посреднический,
	авторитарный  и материнский
\cite[281]{strat}
\end{enumerate}

В типологии стилей воспитания, разработанной Дэном Аллендером, родительская позиция <<да, я
люблю тебя>> --- <<да, ты можешь делать
все что угодно>> соответсвует попустительскому, снисходительному, а в некоторых случаях
безразличному стилю воспитания; <<да, я люблю тебя>>  --- <<нет, ты не можешь делать все что
угодно>> --- авторитетному, посредническому, разрешающему стилю; <<нет, я не люблю тебя>> --- <<да,
ты можешь делать все, что угодно>>  --- игнорирующему, а  <<нет, я не люблю тебя>> --- <<нет, ты не
можешь делать все, что угодно>> ---авторитарному, автократичному.


И.С. Клёцина выделяет следующие типы отцов в зависимости от типа маскулинности:

<Модели отцовского поведения в рамках традиционной модели маскулинности

\begin{itemize}

	\item «Отсутствующий отец» (т. е. отсутствующий, прежде всего в психологическом плане, он может
присутствовать физически, но почти не связан с отцовством);

\item Традиционный отец «старых времен», который заботится о своей семье как руководитель;

Модели отцовского поведения в рамках новой модели маскулинности
	\item «Ответственный отец» активно включен в процесс ухода за детьми и их воспитания, однако
вклад таких
отцов в развитие детей меньше, чем у матерей.
	\item «Новый отец» как развивающийся тип мужчины (пеwfаthеr), который не только берет на себя
ответственность за свою семью, но делит поровну с супругой и домашние обязанности, и обязанности по
уходу за детьми, их развитием и воспитанием>> \cite{clec}.

\end{itemize}

В последней классификации прослеживается историческая динамика стилей отцовства. Можно видеть, как
образ <<хорошего отца>> эволюциронировал от <<отсутствующего>>
до <<ответственного>> и даже до <<нового>>. которого в похожей терминологии можно было бы назвать
<<гиперприсутствующим>>, разделяющим с женщиной на равных ее обязанности по уходу и воспитанию
ребенка, и даже заменяющим ее в этих обязанностях.

В результате этих эволюционных процессов мужчины оказываются в ситуации, требующей от них
исполнения повышенной нагрузки и
реализации высоких и зачастую противоречивых ожиданий. Если некоторым из них удается справляться с
ней, и даже видеть в совмещении мужских и женских ролей путь к самореализации, то отношение других
характеризуется понятием ролевого конфликта. Его суть сводится к тому, что <<от современного мужчины
требуется, чтобы он был сильным и
мягким одновременно. Он должен много работать, но при этом ещё и мириться с работой
жены. Помимо всего того, что традиционно должен уметь и делать мужчина,
предполагается его активная включенность в дела семьи. Многим мужчинам не совсем
понятно, в чем же заключается их роль мужа и отца, и им непросто играть прежнюю роль
кормильца семьи, выполняя при этом еще и домашние и материнские обязанности>> \cite[112]{confl}

<<Ролевой конфликт отражается как во внутриличностной, так и в межличностной
сфере. У мужчины появляются ощущения негативного характера: неудовлетворенность,
тревожность, раздражительность, депрессия, снижение самооценки и стресс>> \cite[113]{confl}.

Как показывают исследования \cite{imaf}, существуют две наиболее типичных возможных выходов
из этого конфликта (необходимо отметить, что они оба не полные и по факту не являются позитивными
исходами ): часть молодых людей
стремятся реализовать этот образ, что приводит к перфекционизму, завышенным требований молодых
людей к себе как к будущим отцам, к своим женам, гиперкотролю: в качестве признаков
готовности к отцовству молодые люди называют успешную карьеру, финансовую независимость,
психологическую и личностную зрелость --- как свою, так и матери ребенка, а также -- чтобы желание
завести ребенка было обоюдным и одновременным \cite{imaf}.

Другая типичная  реакция -- избегание: молодые люди вообще не планируют иметь детей, полагая, --- и
достаточно справедливо, --- что не смогут удовлетворять
тем требованиям, которые общество предъявляет отцам \cite{imaf}.

Можно сказать, что на сегодняшний день очевидна тенденция отхода от классических отцовских ролей в
семье, порисходит их размывание. В то же время новые роли, кторые должны были бы сменить прежние,
находятся пока в стадии формирования  и не обрели ясных очертаний. Ожидания от отцов, перечисленные
выше, -- тесный эмоциональный контакт с ребенком, живое участие в его воспитаниии уходе за ним,
гармоничное сочетание работы и семьи -- относятся скорее к образу идеального отца, оставляя
нерешенными множетсво практических вопросов, с ними связанных.

%Вместе с тем, в центре внимания исследователей оказываются не только детско-родительские отношения.
%Проведенные исследования показывают, что для мужчин важны так же и карьерный рост, и качественный
%полноценный отдых как их индивидуальная сфера \cite{mercoh}.


\section{Отцовство и маскулинность}

Вопросы, связанные с отцовской ролью в семье, не могут быть глубинно раскрыты без
соотнесения их с более широким контекстом --- проблемами маскулинности. Отцовство, как было
отмечено выше, является одной из сфер самоидентификации мужчины, выступая
важной частью его полоролевой саморепрезентации. Проблемы соотношения маскулинности и отцовства
обретают особую актуальность в связи с динамическим характером обоих явлений: <<Культурные и
социальные изменения ослабили связь между маскулинностью и ожиданиями, связанными с ответственным
отцовством. Меняющиеся обстоятельства привели к тому, что взрослые мужчины оказались поделенными
на тех, кто заботится о своих детях, и тех, кто этого не делает>> \cite[132]{f21}. Наблюдаемый в
настоящее время переход от классических к постклассическим моделям маскулинности и отцовства
отражен в исследованиях русскоязычных и зарубежных исследователей. Проблема их соотношения
осмысляется ими в различных вариантах: <<В то время как одни считают, что отцовство утратило
статус нормы и стало добровольным соглашением, другие полагают, что эффективное отцовство является
основным признаком мужественности>> \cite[132]{f21}.

Отцовство может восприниматься мужчинами - участниками проведенных
исследований исследователями либо как важнейшая составляющая
мужественности, без которой она не может быть полностью реализована, либо как нечто, мужественности
противоположное. Многие из них, еще не имея детей, считают, что в будущем станут отцами, и это
станет для них <<подтверждением мужественности>>, <<началом действительно взрослой жизни>>, сделает
частью <<всеобщего продолжения жизни>>, а их жизнь -- <<более серьезной>> \cite[367]{tri}

Образ маскулинности и отцовства не является неизменным. И.С. Клёцина, анализируя
классические и современные модели маскулинности, предполагает, что современные ожидания от отцов
идут вразрез лишь с классичекой моделью маскулинности, в которую входят следующие положения:
<< 1) необходимость отличаться от женщин; 2)
необходимость добиваться успеха и опережать других мужчин; 3) необходимость быть сильным,
независимым и не показывать слабость; 4) необходимость обладать властью над другими>> \cite{clec}.
Сравнение современного видения отцовства может быть представлено как ризис только в случае, если
оно сравнивается с данной моделью.
В то же время они полностью соответсвуют новой -- гендерной -- теории маскулинности: <<Если в рамках
традиционной модели маскулинности в отношениях с детьми отец демонстрирует эмоциональную
дистанцированность, не включенность в повседневные дела детей, властность, строгость и суровость в
оценке их поступков и поведения, то новый мужчина, напротив, проявляет заботу о детях, устанавливает
с ними доверительные и дружеские отношения, находится в постоянном контакте, чтобы понимать проблемы
и интересы ребенка, и быть объективными, но доброжелательным в оценках поступков детей>>
\cite{clec}.

В то же время итальянский исследователь Л.Зойя противопоставляет роли
мужчины и отца, утверждая, что последняя является социальной,
возникшей искусственно на протяжении развития цивилизации. В настоящее время наблюдается
возвращение мужчин от роли отца к роли самца, основной задачей которого является оплодотворение
самки, а не воспитание и обеспечение выживания потомства. Отцовство формируется благодаря
образам (архетипам), присутствующим в коллективном бессознательном общества --- и в
бессознательном отдельных мужчин. Мужчина, становясь отцом, оказывается на пересечении
своего биологического и социального измерения, причем его инстинкты не только не упрощают
выполнение им отцовских функций, но напрямую противоречат им: <<Современный мужчина, даже
если он намеревается быть отцом, с одной стороны, обладает инстинктом, а с другой ---
цивилизованностью, которая хранит обряды, необходимые для отца. Фантазия этого мужчины
обращается к другим женщинам не внутри его цивилизованности, а когда он уходит в отставку
как отец>> \cite[270]{zo}. Процесс возврата мужчин к биологическим функциям Л.Зойя связывает
именно с распадом образа отца и отцовства, вследствие которого <<отцовство не имеет больше
существенного образа. Он зачал физически, но психически не был крещен водой отца>> \cite[270]{zo}.

С другой стороны, изменение ожиданий от отца, изменение его роли --- вместо защитника, добытчика,
олицетворения власти и главы семьи он больше проводит времени с ребенком и заботится о нем
приводит к тому, что, во-первых, мужчина оказывается неспособен открыть для ребенка социальное
измерение жизни, прервать его симбиоз с матерью, а во-вторых, он теряет свою индивидуальность как
мужчина, становится слишком похожим на мать \cite[285]{zo}. Его образ утрачивает
яркость в глазах ребенка. Подобный парадокс еще более усиливает противопоставление биологической
природы мужчины и социальных ожиданий от него: ранее он взял на себя заботу о своем потомстве, что
не свойственно самцам животных, однако его функция социализации ребенка соответствовала его
природе; осуществление же ухода за ребенком, игры  с ним, забота о нем --- это женские функции, не
присущие мужчине.

Парадокс заключается так же и в том, что чаще всего эти функции мужчина выполняет в качестве
помощника матери ребенка, и выполняет их хуже, чем она. Создается ситуация, при
которой отец становится словно бы <<второй матерью>>, причем матерью худшей, чем женщина. В таком
качестве его роль в семье утрачивает свою значимость, и, как следствие --- отпадает
необходимость и в самом его присутствии. <<Отеческая специфика заключается именно в этом: он может
быть с сыном, когда умеет так же носить
броню, может бытьотцом, когда он также воин. В отличие от матери,он не может делать только
что-то одно: Если он будет только носить броню, сын его не узнает; если он не надевает брони
никогда, его не признают как отца>> \cite[287]{zo}.

Ситуация здесь осложняется так же и тем, что далеко не все матери хотят, чтобы их муж (партнер)
осуществлял уход за ребенком и строил с ним близкие, эмоционально-заряженные отношения. В то время
как одни женщины действительно выделяют в качестве качеств <<хорошего отца>>  его вовлеченность;
помощь в уходе за ребенком; любящее и
заботливое отношение к нему, совместные игры с ребенком; финансовая поддержка>> \cite[137]{money},
то другие, наоборот, препятствуют отцу в исполнении подобных функций, мотивируя это его
некомпетентностью, отсутствием необходимых навыков и т.д.


Таким образом, на отца возлагаются обширные и часто противоречивые ожидания: с одной стороны, от
него ожидается большая включенность в процесс ухода за детьми и их воспитания; с другой --- его
роль в материальном обеспечении семьи все еще остается одним из приоритетов, несмотря на растущую
значимость женщины в этой сфере. В психологическом смысле отец призывается, по сути, выстроить свой
собственный симбиоз с ребенком параллельно детско-материнскому, --- и в то же время разорвать его,
ориентируя себя и ребенка на выполнение социальных функций. Так, важнейшими характеристиками
вовлеченного отцовства называют теплоту, чувствительность, близость, нежность и участие отца в
занятиях ребенка \cite[129]{f21}.
Необходимость финансово обеспечивать ребенка с одной стороны, и уделять ему много времени и
внимания -- с другой, таким образом, становятся основными компонентами образа ответственного и
заботливого отца \cite[129]{f21}.



Предполагалось, что гендерная маскулиннсть
в противовес маскулинности традиционной, поможет мужчинам более свободно проявлять себя, проще
выбирать те или иные роли, не зависеть от стереотипов, связанных с образом мужчины в конкретном
обществе. Предполагалось, что такие качества мужчин, как безэмоциональность, стремление к
лидерству, конкурирование между собой, позиция силы, становятся бременем для мужчин,
стирая их индивидуальность. Гендерная маскулинность и связанный с ней новый тип отца -- <<новый
отец>>, однако, рискует превратиться в такой же доминирующий тип маскулинности, отличаясь от
прежней лишь набором черт, которые,  предположительно, присущи мужчине. В силу своей
универсальности, однако, этот тип маскулинности, так же, как и маскулинность традиционная, -
далеко не всегда адекватен психологическим особенностям мужчин, их  потребностям и ожиданиям
от отцовства.  В противовес утверждениям о том, что для мужчины заботиться о ребенке легче,
чем зарабатывать, а также что для ребенка неважно, чья нежность и тепло - главное чтобы они были -
можно привести результаты конкретных исследований, в которых молодые отцы делятся совершенно иными
впечатлениями от отцовтсва. Так, если молодой отец берет отпуск, чтобы провести первые
недели жизни ребенка с ним и женой, то возвращение к работе часто восприниается им как
<<возвращение к нормальному порядку вещей>>, <<лучший выбор для обоих>>, <<облегчение>>
\cite[370]{tri}. Работа описывается ими как <<менее эмоционально утомляющая>> по сравнению с
уходом за ребенком, забота о котором воспринимается как тяжелый труд: <<Нет, я бы сошел с ума, я
искренне уверен, что сошел бы с ума, я не смог бы быть отцом, ухаживающим за ребенком>> (из
интервью) \cite[370-371]{tri}. Многие мужчины в том же исследовании выказывали тревогу относительно
того, как их жены / партнерши будут справляться с ребенком одни, когда они вернутся на работу
\cite[370-371]{tri}.

В то же время, даже отцы, ориентированные на активное участие в жизни детей часто понимают свою
вовлеченность как помощь, которую они будут оказывать
<<по вечерам>>, <<ночью>>, <<в выходные и праздники>>. Помощь эта заключается в основном в том
чтобы <<кормить ребенка>>, <<менять памперсы>>, <<купать>>, <<одевать>>, <<укладывать спать>>,
<<вставать ночью, чтобы проверить, все ли в порядке>> и т.д. \cite[369]{tri}.

Как видно, эти занятия не исключают профессиональной деятельности мужчины, то есть даже мужчины,
ориентированные на вовлеченное и ответственное исполнение отцовской роли, не видят ее в качестве
своего основного занятия, альтернативного работе. При этом необходимо отметить, что обсуждая свое
будущее отцовство, молодые мужчины часто несколько идеализируют свои возможности сочетать
профессиональную деятельность и уход за малышом. После рождения ребенка многие из них бывают
разочароваты и встревожены тем, что не могут уделять ребенку столько времении сил, как они хотели
до его рождения \cite{tri}.



\section{Выводы}

Можно предположить, что формирование новой маскулинности и нового образа отца повлекут за собой
новые сложности в этой сфере. Размывание контраста между отцовскими и материнскими ролями в семье
могут привести к невозможности удовлетворения родителями психологических потребностей ребенка.
Например, с точки зрение психоанализа 	мягкий, эмоциональный отец, устанавливающий симбиотические
отношения с ребенком, не сможет впоследствии конкурировать с ним за мать и способствовать распаду
симбиотической связи между матерью и ребенком, способствовать его самостоятельности. В результате
ребенок может оказаться в ситуации сильной привязанности и психологической зависимости как от
матери, так и от отца и не иметь возможности сепарации.

Кроме того, представляется сомнительным тезис И.С. Клёциной о том, что для ребенка не имеет
существенного значения, кто именно ухаживает за ним, общается с ним, и что симбиоз с мужчиной не
отличается от симбиоза с женщиной \cite{clec}. Возможно, данный тезис справедлив в ситуации, когда
ребенок, находясь в депривации внимания и заботы, оказывается перед выбором --- установить ли
симбиоз с отцом -- или не устанавливать его вообще. Поскольку последний вариант ведет к крайне
тяжелым психотическим расстройствам, можно согласиться, что в такой ситуации первый вариант
предпочтителен. В ситуации, однако, когда ребенок имет равные возможности для получения материнской
или отцовской заботы, представляется более предпочтительным первый вариант: прежде всего, отец
физически не может осуществить всех функций, которые осуществляет мать (речь, прежде всего, о
кормлении грудью -- основе симбиотических отношений); во-вторых, далеко не все мужчины обладают
необходимой эмпатией, навыками и желанием для того, чтобы осуществлять качественную, психологически
комфортную заботу о
ребенке, особенно в первые месяцы жизни.
%Вообще много о стратегиях и стилях воспитания у Орловой.

\chapter{Факторы формирования отцовства}



Психологическая готовность к отцовству – это внутренняя позиция личности,
стержневой образующей которой является целостная система отношений будущего
родителя к отцовству, которая включает отношение к будущему ребенку, себе как
будущему родителю, к родительской роли, а также к родительству в целом \cite[121]{har}. Она
включает в себя  две подсистемы:

\begin{itemize}
	\item тношение к себе (представления о себе как
о мужчине, отце; умения, навыки и деятельность
отца, оценка себя);

\item отношение к ребенку (представления о нем до и во время беременности партнерши, его образ,
сформированный отцом после его рождения, эмоциональное отношение к ребенку, способ взаимодействия с
ним и т.д.)\cite[40]{otage}
\end{itemize}


Формирование представлений мужчины от отцовстве в целом и о собственной отцовской роли в частности
--- это долгий и сложный процесс, на протекание которого влияют многочисленные и разнообразные
факторы. Существует множество классификаций этих фаакторов, которые позволяют ранжировать их по
степени значимости, по тому, являются ли они внутренними (например, психологические особенности
конкретного мужчины) или внешними  (социокультруные представления об отцовстве, принятые в том
или ином обществе, социоэкономическая ситуация в стране и т.д., по характеру их влияния и т.д.
(Кроме внутренних и внешних факторов представляется продуктивным выделение
влияния родительской семьи как своеобразного <<буфера>>, находящегося <<на границе>>
между внутренними и внешними факторами: с одной стороны, стуктура и система отношений в семье
формируются, в том числе, под влиянием <<внешних>> факторов; в то же
время семейная история, структура семья, особенности взаимоотношений ребенка с каждым из
родителей, а так же родителей друг с другом оказывают ключевое влияния на формирование
психологического пространства каждого из ее членов (<<внутреннего фактора>>).

Кроме того, факторы можно разделить на временные --- те, что начинают оказывать влияние по мере
взросления, и постоянные - те, которые присутствуют
постоянно  (общественное мнение, ожидания, коллективное бессознательное). Важно, как эти
факторы преломляются в сознании мальчиков (мужчин) и формируют представления об отцовстве. В этой
связи можно предположить, что есть некий общий набор факторов, которые влияют на формирование
представлений об
отцовстве, но в то же врем нельзя сказать, что действия одного или нескольких таких факторов
непременно приведет к тому или иному видению отцовства. Такие факторы как, например, семейная
история, особенности семейной структуры, сложившиеся паттерны детско-отцовстких отношений,
которые мужчина видел, будучи ребенком, безусловно влияют на его представления об отцовстве, об
идеальном отце и о себе в роли отца; представляется, однако, возможным утверждать, что эти
факторы не являются определяющими, посокльку важную роль играет то, как именно каждый из них
воспринимается конкретным мужчиной \cite[164]{long}.

Важно отметить, что рассуждая о влиянии тех или иных факторов на формирование
представлений об отцовстве, мы имеем дело не с некоей объективной реальностью --- а с восприятием
этой реальности конкретными людьми, в данном случае, мужчинами -- актуальными или потенциальными
отцами.

Очевидно также, что каждая из названных групп факторов, влияющих на формирование представлений
мужчин об отцовстве, включает в себя большое количество отдельных влияний. Рассмотрим каждую из них
более детально.

\section{Внешние факторы}

Про школу --- передают не мужские модели поведения \cite{md}.

В качестве внешних факторов, оказывающих влияние на предтсавление об отцовстве разные исследователи
выделяют следующие:
Д.Росс обществен-
ные или формальные факторы (взаимоотношения
с коллегами по работе; система здравоохранения,
роддома), культурные факторы (культура детства
мальчиков; отношение к отцовской роли; убежде-
ния и ценности семьи, связанные с национальны-
ми особенностями) [3, с. 201–221].\cite[40]{otage}

(у Борисенко – шире – социо-культурный контекст в целом, нормы,
обыча, и, традиции....)\cite{psyot}
Актуальные социокультурные представления об отцовстве, образы <<хорошего>> и <<плохого>>
отца, запечтленные в коллективном бессознательном;
\section{Семейные факторы}

Говоря о влиянии семьи на представление об отцовстве и способы реализации отцовской роли,
необходимо отметить, что, по сути, речь идет о двух семьях: во-первых, это родительская семья, в
которой вырос мужчина, во-вторых, это его собственная семья, которую он создает и в которой
становится отцом. Структура этих семей, способы коммуникации между их членами, убеждения и
ценности, транслирующиеся в них, оказывают одинаково существенное влияние как на теоретичекое и
фантазийное представление от отцовстве, так и на конкретную реализацию отцовских практик. В то же
время влияние этих семей различно по сферам, на которые оно воздействует. Так, в родительской семье
формируется образ отца: мальчик, взаимодействуя со своим отцом, фомирует для себя картину того,
каким должны быть детско-отцовские отношения, какими чертами должен обладать отец; далее, на основе
этих наблюдений мальчик формирует представления о себе как о будущем отце, другими словами -
выстраивает первый фантазийный образ себя как отца, определяет, какими он хотел бы в будущем видеть
отношения с его собственными детьми. Разумеется, на протяжении времени этот внутренний образ
неоднократно подвергается переосмыслению и коррекции, однако влияние родительской модели семьи как
на формирование первоначального образа, так и на его последующие трансформации остается весьма
значительным.
Начальные представления человека о родительстве, в том числе и об отцовстве, напрямую связаны с
качеством отношений, с одной стороны, в
супружеской паре родителей мальчика, а с другой --- с его отношением с каждым из родителей, причем
одни исследователи в этой связи подчеркивают роль отношений ребенка с матерью, а другие --- с
отцом. Например, Ю.В. Борисенко, акцентируя роль матери, подчеркивает, что именно она оказывает
существенное влияние на степень вовлеченности отца в отношения с сыном: <<... существуют
данные, что даже в удовлетворительных брачных отношениях вовлеченность отцов, особенно когда дети
маленькие, часто зависит от отношений и ожиданий матери, ее поддержки отцу, так же как и от степени
ее занятости. [...]  Учитывая мощные культурные ожидания в отношении материнской роли, не
удивительно, что активная отцовская вовлеченность в некотором смысле может угрожать женской
идентичности>> \cite[115]{psyot}.

Другие исследования отношения к отцовству показывают важность отношений мальчика с
отцом. Примером может служить исследование Рейчел Томпсон, показавшее,что позитивное отношение к
собственному отцу коррелирует с желанием мужчины иметь детей и повторять модель отцовства,
существовавшую в родительской семье: обычно те мужчины, которые позитивно высказывались о
своих отцах, стремились повторить эту модель отцовства, а те, кто сохранил негативные
воспоминания об отце, имели намерение исправить его ошибки, сам становясь отцом  \cite[161]{imaf}.
Отцовство для таких мужчин играет центральную роль в жизненных
планах и устремлениях \cite{imaf}.

В то же время, согласно исследованиям В.К. Рахмановой, имеет место и обратная тенденция: <<Новый
взгляд на свои отношения с отцом дает возможность молодому отцу увидеть свои отношения с отцом с
другой стороны, заглянуть в них глубже и уже как бы изнутри, занять рефлексивную позицию по
отношению
к ним. Так, мужчины-отцы, по сравнению с мужчинами, не имеющими детей, чаще говорят о своих
конфликтах с отцами, признают возможность неидеальных отношений с ними, возможность конфликтов,
ссор, запретов и ограничений. Мужчинам, не имеющим опыта отцовства, сравнить это чувство не с чем и
они более идеализированно и стереотипно представляют, какими они будут отцами в будущем.
Следовательно, можно отметить, что опыт отцовства для мужчины позволяет ему во многом принять и,
возможно, разрешить существование трудностей, сложностей в отношениях со своими отцами. Многие
молодые отцы признают, что могут быть строгими, требовательными, могут в чем-то ошибаться, так же
как
ошибались их отцы \cite[54]{relot}.


Создание собственной семьи, беременность жены \ партнерши, рождение первого и последующих детей
становятся важнейшими этапами формирования и воплощения в жизнь реальных отцовских практик. Этот
процесс обусловлен как психологическими особенностями мужчины-отца, так особенностями структуры и
взаимоотношений внутри построенной им семьи.

Таким образом, к семейным факторам формирования представления мужчины оботцовстве можно отнести
следующие:
\begin{enumerate}
	\item Семейная история мужчины: структура его родительсткой семьи,  модель взаимоотношений
между его родителями, формы реализации отцовстких функций его отцом; возможные травмирующие события
(развод родителей, смерть одного из них, эпизоды сексуального или другогонасилия, перенесенные в
детстве и т.п.).
\item <<социокультурные и семейные традиции оформления
родительского
поведения;
этологический фактор раннего контакта ребенка с матерью; особенности общения взрослых членов семьи
между собой>> \cite[16]{varga}>>

\item <<взаимоотношения матери и ребенка
и взаимоотношения отца и ребенка; взаимоотно-
шения мужа и жены; взаимоотношения отец –
мать – ребенок>, \cite[40]{otage}

\item Некоторые исследователи относят  в эту группу факторов так же <<социально-культурный уровень
родителей , уровень дохода семьи, уровень
образования каждого из родителей, их ожидания относительно будущей интеграции
ребенка, социальное окружение семьи, степень открытости семьи, а также тип
взаимодействия супругов>> \cite[286]{strat} и семейную историю: <<раса \ этническая принадлежность
родителей, их образование и финансовое положение>> \cite[164]{long}.

Можно предположить, однако, что в этих критериях важны скорее не сами факты (уровень дохода семьи,
образование родителей и т.д.), -- а их влияние на психологическую обстановку в семье: ее структуру,
способы коммуникации между ее членами, эмоциональный фон отношений в ней и т.д. При этом нужно
отметить, что не существует однозначной прямой взаимосвязи между неблагоприятными объективными
показателями (например, семья с низким уровнем дохода, родители не имеют высшего образования и
т.п.) и психологическим неблагополучием семьи, равно как и <<благоприятные>> объективные факторы не
 обязательно говорят о психологическом здоровье семьи.

 \item Структура семьи, созданной мужчиной (наличие/отсутствие оформления брака,
очередность брака, количество, пол и возраст детей, их психологические, физические и
поведенческие особенности, является ли семья многопоколенной (в которой бабушки и дедушки
проживают вместе с детьми или поддерживают с ними постоянный тесный контакт, играя
значительную роль в их жизни); особенности коммуникации в семейной паре; отношение жены \ партнерши
к избранному мужчиной стилю отцовства, степень удовлетворенности отношениями и т.д.

\item К этой же группе факторов можно отнести и взаимоотношения семьи (как родительской, так и
созданной мужчиной) с ближайшим окружением --- взаимоотношения с родственниками, соседями, друзьями
и т.д. \cite[40]{otage}. Этот показатель гворит, в том числе, о степени открытости семьи как
системы, об умении и желании ее членов коммуницировать с окружающим миром, с людьми, не входящими в
систему и, возможно, обладающими отличными от присущих ей системами ценностей, мировоззрением,
социально-экономическим статусом, психологическими характеристиками и т.д. Высокая степень
открытости семьи, в сочетании с умением экологично выстраивать собственные границы и сохранять
наиболее важные свои характеристики могут стать важнейшей основой для формирования характера
ее членов и способствовать формирваонию непротиворечивых, гибких и аутентичных представлений о
родиельских ролях, в том числе, об отцовской роли.

\end{enumerate}




\section{Психологические (внутренние) факторы}

К внутренним факторам, оказывающим влияние на формирование отцовства, можно отнести  особенности
психологического пространства мужчины. Поскольку отцовство -- это, прежде всего, построение
реальных отношений мужчины с ребенком (детьми), то в самом широком смысле комопнентами этого
психологического простарнства могут стать отношение мужчины к самому себе (его самооценка как отца,
человека, профессионала и т.п.), и его отношение к ребенку. Заметим, что, поскольку взаимоотношения
между отцом и ребенком взаимонаправлены, в качестве психологического их измерения стоило бы
рассатривать и отношения ребенка к отцу и к самому себе, однако в рамках данной работы мы
рассматриваем отцовство лишь с точки зрения мужчины, как его экзистенциальный опыт, и потому
ограничимся лишь его восприятием этих отношений.

Становление мужчины отцом не происходит мгновенно. Этот процесс можно условно разделить на
несколько этапов.

\begin{enumerate}
	\item Формирование первоначальных предтсавлений об отцовстве. Этот период длиться с детства,
когда ребенок начинает общаться со своими родителями, в частности, отцом, и формировать свои
преставления о том, какими должны быть эти отношения. Можно сказать, что это период теоретических
знаний и предтавлений об отцовстве, когда ребенок (молодой человек) лишь гипотетички <<примеряет>>
на себя роль отца.

\item Беременность жены. Весьма важный период, для многих мужчин  оказываюийся даже более
стрессовым, чем само рождение ребенка и постнатальный период  \cite[313]{flit}.

\item Роды выделяются в отдельный период, поскольку провоцируют сильные психологические переживания
у отцов.

\item Постнатальный период, который в свою очередь может разделяться на ранний и поздний, в
зависимости от времени, прошедшего с момента рождения ребенка.

\end{enumerate}

Разумеется, развитие мужчины в качестве отца не прекращается на этом. Отцовство, будучи одной из
важнейшей сфер мужской самоидентификации, предполагает постоянное развитие и изменение параллельно
с развитием личности мужчины, формированием системы взаимоотношений в его семье, изменением
количества детей в семье и их возраста и т.д.

Переживания, с которыми сталкиваются мужчины на каждом из вышеперечисленных периодов, влияютна их
выбор той или иной стратегии отцоской стратегии, того или иного стиля отцовства. Выше мы
рассмотрели факторы, формирующие отношение человека к отцовству на первом этапе.  В настоящем
разделе представляется важным рассотреть эмоциональные реакции мужчин на беременность их жен и
рождение ребенка.

Необходимо отметить, что на каждом из этих этапов мужчины переживают весьма сильные и сложные
чувства и состояния, где позитивные эмоции смешиваются с негативными.

\subsection{Беременность}

Беременность становится этапом, на котором происходит психологическая реорганизация мужчины.
Сформированный им ранее образ себя как отца начинает восприниматься не как гипотетичская
возможность, а как часть реальности, которая в скором времени получит физическое воплощение. При
этом многие позиции и установки, касающиеся отцовства и распределения ролей в семье подвергаются
корректировке. Именно поэтому беременность супруги становится одним из наиболее стрессовых периодов
в переходе мужчины к роли отца. process (23,25,26). Как показывают исследования, наибольшие
сложности для мужчин в этот период представляет необходимость ограничить свою независимость,
научиться новому взгляду на жизнь, принятие того, что начинается новая фаза их жизни, а также
умение  переживать чувства бессилия и отсутствия контроля над ситуацией. Эти задачи порождают
сложное  психологическое состояние, которое может выражаться в мрачности, раздражительности,
тревожности, фрустрации и негативном восприятии себя  \cite[314]{flit}


Даже те мужчины, которые заинтересованы в активном участии в воспитании ребенка и уходе за ним,
испытывают сложности во время беременности партнерши, поскольку это сталкивается с их
представлениями о равных гендерных ролях. Этот период становится особенно проблематичным для
мужчин, относящихся к типу <<new father>>, т.к. бросает вызов их представлениям об абсолютном
равенстве гендерных ролей \cite[20]{long}.

Роды, на которых мужчины все чаще присутствуют, становятся в этом плане настоящим испытанием
для них, принося множетсво негативных чувств: <<Проведенные исследования чувств, которые
испытывали отцы, присутствовавшие на родах,показывают, что часто эти чувствами были
беспомощность, ощущение собственной бесполезности и раздражение. Часто отцы говорят о том,
что не ожиали, что присутствие на родах будет настолько истощающим, они чувствовали свое
присутствие тем неуместным, а себя -- беззащитными, неготовыми к этому, нуждающимися в
психологической помощи>> \cite[313]{flit}.

С другой стороны, присутствие на родах может играть и позитивную роль, помогая мужчинам более
быстро свыкнуться со статусом отца, поскольку, помимо вышеперечисленных отрицательных чувств,
мужчины часто испытывают удовлетворение и гордость, наблюдая за рожденим своего ребенка
\cite[314]{flit}.







%Although we had expected childbirth to be the most
%s%tressful time for men, the studies we examined indicated
%pregnancy as being the most stressful period (30,33–35).
%This finding could be because men must undergo psycho-
%logical reorganization during this time (33) when they
%may also be aiming to adhere to a father image based
%on close involvement with family and childcare—an
%image for which they never inherited a role model (36).
 %\cite[313]{flit}.







\subsection{Постнатальный период}



Позже эти же мужчины чувуствут себя исключенными
из-за грудного вскармливания \cite[21]{long}. В результате отцы помещают свои намерения быть
вовлеченными в жизнь детей в будущее, когда дети подрастут и уже не будут полностью физически
зависеть от матери \cite[22]{long}.


После рождения ребенка главной задачей отца становится осуществление на практике того образа
собственного отцовства, который они сформировали для себя в течение беременности жены. Разумеется,
этот образ, столкнувшись с действительностью, может подвергаться всевозможным более или менее
значительным коррекциямй. Та или иная выбранная мужчиной модель отцовства может не отвечать реально
складывающейся ситуации либо не соответствовать психологическомускладу самого отца (особенно
вероятно это в случае, когда та или иная модель выбиралась мужчиной, исходя не из реального, а
идеального своего Я). Это демонстрирует глубинную взаимосвязь между образом себя как отца, стпени
удовлетворенности своим реальным отцовством и качеством осуществления
отцовских функций: <<Отцы, которым удалось выстроить позитивный, относительно не противоречивый
образ себя как отца, были сильно мотивированы к еще более глубокой вовлеченности в жизнь их
детей, в то время как тем, чей образ себя как отца содержал много противоречащих друг другу
элементов, было сложно справляться с требованиями, связанными с выполнением отцовсткой роли, и
они часто чувствовали себя исключенными из из семейных отношений>> \cite[314]{flit}.

Одним из прикладных аспектов выстраивания отношений отца с ребенком становится вопрос сочетания
профессиональной деятельности отца и связанного с ней финансового обеспечения семьи --- и
вовлеченности мужчины в уход за ребенком,  воспитание и общение с ним.

По мере того как ребенок растет и перестает полностью физически зависеть от матери, для отцов
открывается больше возможностей быть вовлеченными в общение, воспитание и уход за ребенком.
Лонгитюдные исследования, проводившиеся в течение 8 лет в включавшие в себя интервью с отцами
в период беременности их партнерш, сразу после рождения детей и через восемь лет после этого,
показали, что часто они используют эти возможности и устанавливают с детьми тесные, теплые
отношения. Парадоксально, однако, что эта возросшая близость не приносит отцам удовлетворения: с
одной стороны, они по-прежнему считают, что вовлечены в жизнь детей недостаточно и снова переносят
свои ожидания большей вовлеченности в будущее, когда дети станут еще старше. С другой стороны,
поддержка постоянно тесных отношений с детьми воспринимаются отцами как
психологически-изматывающая, требующая постоянного напряжения. Именно поэтому наряду с надеждой
на более тесное общение с детьми в будущем, отцы досаточно часто
ждут когда дети станут взрослыми и не будут так много требовать от них \cite[22]{long}.

В целом, отцовство становится сложным и во многом противоречивым экзстенциальным опытом для
мужчины.
 С одной стороны, это важная ценность, одна из
значимых сфер самоидентификации - а с другой из-за полоролевого конфликта  мужчина попадает в
ситуацию ролевого стресса, так как не может
качественно выполнять требования новой ситуации, вследствие чего происходит
ослабление власти в семье, появление субъективного ощущения отцовской
некомпетентности, следствием которого является появление у мужчины-отца
субъективных ощущений негативного характера: неудовлетворенности, тревоги,
раздражительности, обидчивости \cite[111]{confl}.


К психологическим факторам формирования представлений от отцовстве исследователи относят:

\begin{enumerate}
\item  физиологические и психологические особенности конкретного мужчины :уровень психического
функционирования,
уровень интеллектуального и эмоционального развития, тип личности, темперамент, уровень самооценки
и т.д.

\item Особенности личности отца: система ценностей, вероисповедание, социальная
успешность, <<отношения, убеждения и мотивация отца; вза-
имоотношения с семьей, в которой вырос; возраст,
когда принял родительскую роль>> \cite[40]{otage}

\item Индивидуальные особенности ребенка -- пол, возраст, порядок появления в семье, уровень
психического и физического развития, характер и т.д.

\end{enumerate}


%Именно отражение влияния первых трех групп факторов в сознании мужчины позволяет говорить о них как
%о психологических факторах формирования отцовства. Кроме того, именно они определяют степень и
%качество влияния на него других вышеперечисленных факторов: идентичные ситуации могут оказывать
%различное влияние на различных людей.
%\end{enumerate}

Очевидно, что рассмотренные выше группы факторов не существуют изолированного друг от друга --- они
находятся в теснейшей взаимосвязи, оказывая сильное влияние друг на друга. <<Личностные черты
родителя
(общительность, гармоничность душевного мира, открытость, способность к
изменениям и анализу, невротичность и другие) являются важной предпосылкой
формирования и проявления родительского отношения. Родительская семья задает
молодому человеку определенную модель, образец будущей семьи или ее анти-идеал.
Общественный уровень влияний задает для мужчин «отправную точку», определенный
образец отцовства, который, преломленный через особенности личности, ценностно-
мотивационную сферу, а также опыт, полученный в родительской семье, дает начало
формированию отцовства в каждом конкретном случае. Совокупное воздействие
факторов приводит к формированию определенной модели отцовства в каждом
конкретном случае \cite[122]{har}

Отдельный пласт исследований посвящен специфическим случаям отцовства, когда личностные факторы
отца могут существенно повлиять на восприятие и исполнение им отцовских функций. К таким
исследованиям относится, например, изучение восприятия отцовства мужчинами, пережившими сексуальное
насилие в детстве \cite{sex}, мужчин с психическими заболеваниями \cite{gbi}, алкогольной
зависимостью и агрессивным поведением \cite{alc}, а также мужчин-геев \cite{gay}.

Сексуальное насилие, пережитое мальчиком в детстве, оказывает сильнейшее влияние на формирование
его мужественности в целом и в частности его представления об отцовстве. В целом мужчины,
пережившие сексуальное насилие в детстве, испытывают больше страхов в отношении своего возможного
отцовства.  Что касается образа себя как отца, то они склонны видеть себя гиперопекающими  и
гиперконтролирующими по отношению к своим детям, чтобы предотватить возможность повторения
ситуации насилия с ними. С другой стороны, некоторые мужчины, пережившие насилие, предполагают,
что они будут достаточно отстраненными отцами ввиду сложностей с эмоциональным и физическим
контактом с детьми, порождаемыми их травмой. В этой связи присутствует так же страх, что отцовство
станет катализатором ретравматизации.

Возможность повторения ситуации насилия уже с собственными детьми пугает часть мужчин-жертв
насилия, однако более выражен другой страх --- того, что в общественном мнении жертва насилия
со временем становится насильником. Мужчины опасаются, что именно сквозь эту призму их будут
воспринимать окружающие, даже если реальной склонностью к насилию они не обладают \cite{sex}.

Характер исследований так же разнообразен, как и охватываемые ими вопросы. Проводятся как
исследования конкретных узких вопросов, так и более масштабные лонгитюдные исследования. Примером
может служить исследования восприятия отцами своих детей и своего нового положения в течение срока
беременности и первых полутора лет жизни ребенка \cite{percep}. Нужно отметить, что, несмотря на
изменяющиеся ожидания от отцовства (а возможно,именно по этой причине), --- эмоциональное состояние
отцов в этот период скорее пониженное: эмоциональными реакциями на эту фазу часто
становятся растерянность, смятение и удивление \cite[12]{meta}>>.

\chapter{Выводы}

Круг психологических факторов, влияющих на то, каким
образом человек воспринимает и реализует свои отцовские функции, весьма широк, однако
представляется
возможным выделить следующие наиболее существенные из них.




Годфруа Ж 
И.С. Кон Этнография родительства
ИС Кон -
,
В.Ф. Анурин, мерещакова verlinden m
т.в. архирева
р.в. овчарова
laRossa r reitzes d  - о социокультурной стороне О.
Все говорят о том, что сейчас очень высоки требования к вовлечености отца в жизнь детей. Уровень ее 
фактически приблежается к материнскому. То есть от функций защитника, кормильца, носителя власти, 
примера для подражания – к болле личностному восприятию.

Крайг г – о в разных культурах.



с.79 – структура отцовства
найти
2 подхода к отцовству – 1. исходяот ребенка,
2. от личности родителя \cite[11]{psyot}, нам надо второй
с 11, мы в рамках данного подхода будем говорить о том, какие факторы влияют на реализацию 
отцовства.
Орисенко, с.35 – внешние и внуиренние факторы ф ыормирования личности. На формирование отцовства так 
же влияют внутренние и внешние факторы \cite[35]{psyot}.
3. его взаимоотношения с его отцом \cite[49]{rah}


становление отцом активизирует влияние всех этих факторов.

методики там же




\printbibliography[env=gostbibliography,sorting=ntvy]
\addcontentsline{toc}{chapter}{Литература}

\end{document}

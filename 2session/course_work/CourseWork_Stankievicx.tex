\documentclass{../../common/thesisbyxetex}

\usepackage{enumitem}
\setlist{nolistsep}	% отступы между элементами перечесления

\addbibresource{../../common/bibliography.bib}

\begin{document}

\hypersetup{
pdftitle = {Психологические факторы формирования отцовства},
pdfauthor = {Станкевич Наталия Александровна},
pdfsubject = {курсовая работа},
pdfkeywords = {отцовство, стили, курсова}
}% End of hypersetup
 
 Психологические факторы формирования отцовства
Станкевич Наталия Александровна,
психология, 1 курс, второе высшее
Введение
Объект исследования – феномен отцовства
Предмет – психологические факторы формирования стиля отцовства
Цель – проанализировать влияние различных психологических факторов на формирование отношения мужчин 
к роли отца и выбора ими тех или иных стилей отцовства. 
Достижение поставленной цели предполагает решение следующих задач:
1. Дать определение понятию “стиль отцовства” и выстроить классификацию стилей.
2. Выявить основные психологические факторы, оказывающие влияние на формирование стиля отцовства.
3. Проанализировать влияние этих факторов на формирование стиля отцовства.


\section*{Введение} 

Цель – проанализировать психологические факторы формирования отцовства.
Объект исследования – феномен отцовства в психологии
Предмет – психологические факторы формирования отцовства
Достижение поставленной цели предполагает решение следующих задач:
1. Дать определение понятию “отцовство”, раскрыть структуру отцовства как психологического феномена.
откуда Вы взяли этот стиль отцовства? и зачем он Вам?.
2.Определить историческую и социокультурную специфику отцовства.
3.Выявить основные психологические факторы формирования отцовства.


Актуальность темы исследования--- что нет комплексных исследований того, что влияет на становление
отцом, на
формирование образа отцовства who will become fathers in the future. No researcher
has described the paths that move boys to the practice
of fathering, nor even constructed a uniÞed theory ex-
plaining the complex set of developmental processes
that give meaning to and shape the practice of father-
hood (Lamb, 1997; Tanfer \& Mott, 1998). Similarly, rel-
atively little is known about factors that contribute to
changes in a fatherÕs involvement in his childÕs life
over time. We need theories that articulate relations
between the timing of fatherhood in relation to its
own course and the course of childrenÕs development
(Neville \& Parke, 1997) \cite[131]{f21}.




Трансформации, затронувшие институт семьи в последней трети ХХ - начале ХХI вв.,  спровоцировали 
широкую научную дискуссию, в фокусе которой оказались такие проблемы как форма семейных 
отношений \cite{gay, legfat}, фактическое содержание мужских и женских ролей \cite{mercoh, percep}, 
новые тенденции в детско-родительских отношения и отношениях в паре \cite{relot, sex}.

Одной из наиболее значимых тем в этой дискуссии становится тема родительства. Традиционно при ее 
обсуждении особое внимание уделялось матери: ее отношениям с детьми разного возраста и разного 
пола \cite{maler}, возможности совмещения материнства и профессиональной самореализации, 
социокультурным особенностям реализации материнской роли и т.д. В последнее время, однако, 
размывание гендерных ролей и серьезные изменения в организации и функционировании семьи 
вывели на первый план вопросы, связанные с ролью мужчины в этих процессах. Отцовство, будучи одной 
из ключевых сфер самоидентификации мужчины \cite{imaf}, приобретает в этой связи особую 
актуальность.

Исследования феномена отцовства могут ведутся  в различных плоскостях: можно говорить о 
психологическом, социальном, культурном измерении, о представлениях об отцовстве мужчин и ожиданиях 
и представлениях женщин о <<хорошем отце>>.

Российский исследователь И.С. Кон указывал, что отцовство может быть исследовано как <<социальный 
институт, то, как его представляет себе общество, [и] отцовство как деятельность, практики и 
стили поведения. Для исследования этих явлений нужны разные источники и методы. В первом случае 
осуществляется реконструкция и анализ социокультурных норм, чего общество ожидает от отца "вообще". 
Во втором происходит описание и анализ того, что фактически делают и чувствуют конкретные отцы, 
какова психология отцовства \cite[3]{konot}.

Созвучна этому и мысль Ю.В. Борисенко, который выделяет два подхода к изучению родительства и, в 
частности, отцовства: <<Существует два подхода к изучению родительства, в зависимости от того, кто 
считается отправной точкой изучения - ребенок или родитель. Первый, наиболее распространенный, 
подход рассматривает родительство применительно к развитию ребенка [...], во втором подходе 
рассматривается выполнение родительской роли через призму личности родителя. Здесь исследуется 
самореализация личности в родительстве, вводятся понятия «социальная роль», «статус», «социальные 
нормы», «стереотипы и требования», исследуется феномен так называемого родительского «инстинкта»
(материнского и отцовского), исследуются чувства, образы - Я, Я-концепция и другие личностные 
характеристики, так или иначе связанные и изменяющиеся с родительством \cite[11]{psyot}

В соответствии с выбранной темой курсовой работы, интерес для нас будет представлять массив 
литературы --- как теоретической, так и описывающей конкретные исследования, --- относящийся ко 
второму направлению в этой классификации, а именно к психологическому измерению отцовства. 
Социокультурные же нормы, ожидания общества мужчин в роли отцов, стереотипы и т.д. будут 
рассматриваться нами лишь в той степени, в которой они влияют на психологию отцов и конкретные 
методы реализации ими отцовских функций.

Исследования отцовства у многих авторов, как русскоязычных \cite{relot, psyot}, так и зарубежных 
\cite{meta, morfat, legfat}, начинаются с констатации факта глубинных изменений, которые происходят 
в понимании и реализации мужчинами своей роли в качестве отцов. Если традиционное понимание 
отцовской роли сводилось к отцу-<<добытчику>>, <<примеру для подражания>>, а так же 
<<персонификации власти>> в отношении детей, --- то новое видение подразумевает гораздо большую 
вовлеченность отца в процесс воспитания. В эмоциональном измерении это выражается в  эмоциональной 
ангажированности отца в процессы воспитания и взаимодействия с детьми, его близости, открытости и 
доступности для ребенка. Процессуальное измерение нового типа отцовства включат такие параметры как 
внимание, которое отец уделяет ребенку, их общие занятия, время, проведенное вместе и т.д. 
Так, исследования проведенные среди молодых матерей показали, что существенными  для них 
характеристиками <<хорошего отца>> являются: вовлеченность; помощь в уходе за ребенком; любящее и 
заботливое отношение к нему, совместные игры с ребенком; финансовая поддержка>> \cite[137]{money}.

Касательно этих тенденций исследования отцовства могут быть сгруппированы на основании той оценки, 
которую их авторы им дают: если одни воспринимают их как кризис семьи, а <<вовлеченное отцовство>> 
как вызов, то другие, как, например, И.С. Кон, предполагают, что изменение роли мужчин в семейных 
отношениях --- это часть естественного процесса размывания гендерных ролей и ослабления 
традиционного противопоставления между ними \cite{konmen}

Необходимо отметить, что подход, ориентированный на психологические факторы формирования и 
реализации отцовства, включает в себя широкий и разнообразный спектр более узких вопросов, 
исследования которых отражены в различных теоретических статьях и практических 
социально-психологических исследованиях. 

Круг психологических факторов, влияющих на то, каким 
образом человек воспринимает и реализует свои отцовские функции, весьма широк, однако 
представляется 
возможным выделить следующие наиболее существенные из них.



\section{Изменения структуры и функций института семьи}




\section{Семейная история}


Clearly, the meaning and practices of fatherhood
are related to gender identity (Daly, 1993; Lytton \&
Romney, 1991; Witt, 1997) and to menÕs experiences
with their own fathers and other kin (Cowan \& Cowan,
\cite[131]{f21}

Начальные представления человека о родительстве, в том числе и об отцовстве, начинают формироваться 
еще в детстве, в родительской семье, и напрямую связаны с качеством отношений, с одной стороны, в 
супружеской паре родителей мальчика, а с другой --- с его отношением с каждым из родителей, причем 
одни исследователи в этой связи подчеркивают роль отношений ребенка с матерью, а другие --- с 
отцом. Например, Ю.В. Борисенко, акцентируя роль матери, подчеркивает, что именно она оказывает 
существенное влияние на степень вовлеченности отца в отношения с сыном: <<... существуют 
данные, что даже в удовлетворительных брачных отношениях вовлеченность отцов, особенно когда дети 
маленькие, часто зависит от отношений и ожиданий матери, ее поддержки отцу, так же как и от степени 
ее занятости. [...]  Учитывая мощные культурные ожидания в отношении материнской роли, не 
удивительно, что активная отцовская вовлеченность в некотором смысле может угрожать женской 
идентичности>> \cite[115]{psyot}.

Другие исследования отношения к отцовству показывают важность отношений мальчика с 
отцом. Примером может служить исследование Рейчел Томпсон, показавшее,что позитивное отношение к 
собственному отцу коррелирует с желанием мужчины иметь детей и повторять модель отцовства, 
существовавшую в родительской семье: обычно те мужчины, которые позитивно высказывались о 
своих отцах, стремились повторить эту модель отцовства, а те, кто сохранил негативные 
воспоминания об отце, имели намерение исправить его ошибки, сам становясь отцом  \cite[161]{imaf}.  
Отцовство для таких мужчин играет центральную роль в жизненных 
планах и устремлениях \cite{imaf}. 

В то же время, согласно исследованиям В.К. Рахмановой, имеет место и обратная тенденция: <<Новый 
взгляд на свои отношения с отцом дает возможность молодому отцу увидеть свои отношения с отцом с 
другой стороны, заглянуть в них глубже и уже как бы изнутри, занять рефлексивную позицию по 
отношению 
к ним. Так, мужчины-отцы, по сравнению с мужчинами, не имеющими детей, чаще говорят о своих 
конфликтах с отцами, признают возможность неидеальных отношений с ними, возможность конфликтов, 
ссор, запретов и ограничений. Мужчинам, не имеющим опыта отцовства, сравнить это чувство не с чем и 
они более идеализированно и стереотипно представляют, какими они будут отцами в будущем. 
Следовательно, можно отметить, что опыт отцовства для мужчины позволяет ему во многом принять и, 
возможно, разрешить существование трудностей, сложностей в отношениях со своими отцами. Многие 
молодые отцы признают, что могут быть строгими, требовательными, могут в чем-то ошибаться, так же 
как 
ошибались их отцы \cite[54]{relot}.

\section{Личностные особенности} 

Отдельный пласт исследований посвящен специфическим случаям отцовства, когда личностные факторы 
отца могут существенно повлиять на восприятие и исполнение им отцовских функций. К таким 
исследованиям относится, например, изучение восприятия отцовства мужчинами, пережившими сексуальное 
насилие в детстве \cite{sex}, мужчин с психическими заболеваниями \cite{gbi}, алкогольной 
зависимостью и агрессивным поведением \cite{alc}, а также мужчин-геев \cite{gay}.

Рост количества рождений детей вне брака.

Про школу --- передают не мужские модели поведения \cite{md}.


Вообще факторы --- временные --- те что по мере взросления появляются, и социальные - те, что 
постоянно есть (общественное мнение, ожидания, коллективное бессознательное). Важно, как эти 
факторы преломляются в сознании мальчиков (мужчин) и формируют представления об отцовстве. В этой 
свзи можно предположить, что есть некий общий набор факторов, которые влияют на формировани 
отцовства, но в то же врем нельзя сказать, что действия одного или нескольких таких факторов 
непременно приведет к тому или иному видению отцовства. Такие факторы как, например, семейная 
история, особенности семейной структуры, сложившиеся паттерны детско-отцовстких отношений, 
которые мужчина видел, будучи ребенком, безусловно влияют на его представления об отцовстве, об 
идеальном отце и о себе в роли отца; представляется, однако, возможным утверждать, что эти 
факторы не являются определяющими, посокльку важную роль играет то, как именно каждый из них 
воспринимается конкретным мужчиной \cite[164]{long}. 

\cite[164]{long}

Сексуальное насилие и отцовство

Сексуальное насилие, пережитое мальчиком в детстве, оказывает сильнейшее влияние на формирование 
его мужественности в целом и в частности его представления об отцовстве. В целом мужчины, 
пережившие сексуальное насилие в детстве, испытывают больше страхов в отношении своего возможного 
отцовства.  Что касается образа себя как отца, то они склонны видеть себя гиперопекающими  и 
гиперконтролирующими по отношению к своим детям, чтобы предотватить возможность повторения 
ситуации насилия с ними. С другой стороны, некоторые мужчины, пережившие насилие, предполагают, 
что они будут достаточно отстраненными отцами ввиду сложностей с эмоциональным и физическим 
контактом с детьми, порождаемыми их травмой. В этой связи присутствует так же страх, что отцовство 
станет катализатором ретравматизации.

Возможность повторения ситуации насилия уже с собственными детьми пугает часть мужчин-жертв 
насилия, однако более выражен другой страх --- того, что в общественном мнении жертва насилия 
со временем становится насильником. Мужчины опасаются, что именно сквозь эту призму их будут 
воспринимать окружающие, даже если реальной склонностью к насилию они не обладают \cite{sex}. 

Характер исследований так же разнообразен, как и охватываемые ими вопросы. Проводятся как 
исследования конкретных узких вопросов, так и более масштабные лонгитюдные исследования. Примером 
может служить исследования восприятия отцами своих детей и своего нового положения в течение срока 
беременности и первых полутора лет жизни ребенка \cite{percep}. Нужно отметить, что, несмотря на 
изменяющиеся ожидания от отцовства (а возможно,именно по этой причине), --- эмоциональное состояние 
отцов в этот период скорее пониженное: эмоциональными реакциями на эту фазу часто 
становятся растерянность, смятение и удивление \cite[12]{meta}>>.

\section{Основные понятия}

Формирование отцовства --- это процесс формирования представлений мужчины о себе в качестве отца, в 
который можно включать такие компоненты как его представление о самом себе (как личности, мужчине, 
профессионале и тп.), представления об идеальном отце --- каким он должен быть, каковы наиболее 
важные его качества, а так же о себе в этой роли --- каким отцом должен быть я. Сюда же может 
включаться представления о себе как ореальном отце (если уже есть дети), а так же анализ различных 
сложностей в реализации отцовских функций --- настоящие или предполагаемые. 
Одним словом, это формирование представлений об идеальном отце и формирования образа себя как отца.

Фактическими критериями готовности  к отцовству


\subsection{Оценка тенденций}






\section{Структура отцовства.}


Содержанием второго подхода к исследованию «Отцовства» как фак-
тора развития современного мужчины являются следующие аспекты:
– «Отцовство» как совокупность свойств сознания человека, таких
как: потребности, влечения, желания, установки, ценностные ориентации,
мировоззрение, «образ Я», «Я-концепция»;
– «Отцовство» как принятие социальных требований к отцу, приня-
http://ej.kubagro.ru/2013/02/pdf/19.pdf
Научный журнал КубГАУ, No86(02), 2013 года
 5
тие социальной роли отца, принятие статуса отца в социуме, принятия сво-
их отцовских чувств;
– «Отцовство» как проблема развития самосознания мужчины, про-
блема самооценки и самоконтроля, развития осознания себя отцом, необ-
ходимости соблюдения социальных норм, собственной значимости для ре-
бенка \cite{psyot}



\subsection{Роли отца в структуре семьи.}


Таким образом, в современной ситуации отцовства можно выделить две значительные, противоположные
тенденции. С одной стороны  в отцовстве наблюдаются разрушительные тенденции: растет количество
разводов и, соответсвенно, неполных семей, дети в которых растут, не поддерживая значимых отношений
с отцами. Согласно проведенным исследованиям, лишь небольшой процент отцов, дети которых
родились вне брака, продолжали жить совместно с ними и их матерями либо поддерживать с ними близкие
отношения спустя пять лет \cite{long}. В результате появляется все больше мужчин, не имеющих
отцовского образа, не имевших возможности сформировать для себя позитивную модель мужественности и
получить пример детско-отцовских отношений. Эта ситуация весьма актуальна на современном
постсоветском пространстве.



Все это позволяет исследователям говорить о кризисе традиционной семьи и, в частности, отцовства.
Необходимо, однако, отметить, что в синергетическом понимании кризис представляет собой некую точку
бифуркации, преодоление которой предполагает либо разрушение системы, либо вывод ее на качественно
иной уровень функционирования, предполагающий новые условия и законы развития и новое состояние
системы в целом.

Одним из факторов, оказывающим существенное влияние на представления об отцовстве, являются
изменения в структуре и функциях семьи. Помимо очевидных изменений в гендерных ролях, о которых шла
речь выше (мужчины больше вовлекаются в воспитание детей, женщины больше внимания уделяют
работе,
карьере и т.д.) --- меняется само видение семьи. Эти изменения можно описать следующими
тенденциями: во-первых, меняется конфигурация семьи. На смену нуклеарной семье приходит расширенные
ее модели --- семьи с одним родителем, бездетные семьи, семьи с однополыми родителями и т.д.
Понимание отцовства
в этой связи также претерпевает изменения, отходя от классической модели, в которой процедура
установлени отцовства основывалась на предположении о том. что отцом ребенка является, скорее всего
 тот, кто имел сексуальные отношения с его матерью, а этим человеком, вероятнее всего,
является муж \cite[318]{legfat}

Во-вторых, происходит переход от понимания семьи как структурной единицы --- <<ячейки общества>>,
--- к семье как процессу. На первое место выдвигаются не элементы семьи, субсистемы семейной
системы, --- а отношения между ее членами, которые и создают то, что называется <<быть семьей>>
\cite{fam}. В этой связи состав семьи --- возраст, пол и количество ее членов уже не так
существенны для того, чтобы определить какую-либо группу людей как семью. Очевидно, что такое
понимание семьи открывает новые перспективы в видении детско-родительских, в том числе
отцовстко-детских отношений. Акцент в них делается также на отношениях, на поддержании аффективной
связи между отцом и ребенком, что, несомненно,  вписывается в модель <<вовлеченного отцовства>>.
Хотя вместе с тем, такое изменение отцовской роли, участие отца в родах и последующем уходе за
ребенком представляется некоторым исследователям
революционным \cite[15]{fatpsy}.

Вместе с тем, в центре внимания исследователей оказываются не только детско-родительские отношения.
Проведенные исследования показывают, что для мужчин важны так же и карьерный рост, и качественный
полноценный отдых как их индивидуальная сфера \cite{mercoh}.

В связи с этим мы можем наблюдать новые тенденции развития отцовства, которые приходят на смену
прежнему его пониманию. Прежде всего в этой связи необходимо упомянуть радикальное изменение
социокультурных представлений об отцовстве и, соответсвенно, формирование нового образа отца, новых
ожиданий от отцов. В этой связи необходимо упомянуть такие понятия как <<вовлеченное отцовство>>,
при котором мужчина принимает активное участие в уходе за детьми и их воспитании, а так же
новый,  только формирующийся и пока не получивший широкого распространения  тип <<нового
отцовства>> (<<new fatherhood>>), когда мужчина полностью разделяет домашние обязанности и хлопоты
по уходу за детьми со своей женой, или даже полностью берет на себя эту роль, оставляя работу, в то
время как фунцию материального и финансового обеспечения семьи берет на себя женщина.

%эта вторая тенденция приводит к разговору о маскулинности и отцовстве, может вставить этот раздел
%сюда. Говорят, что и это кризис - кризис маскулинности. Утверждения, что забота о ребенке легче,
%чем зарабатывать, а также что для ребенка неважно, чья нежность и тепло - главное чтобы они были -
%оочень сомнительны.
to remain at home (through the initial months at least) caring for the child. Returning
to work is described in various ways by the new fathers. For some it signals a return
“to normalness,” “the best of both worlds,” “a relief” and “a release.” Feelings of “guilt”
371
MILLER
are also expressed by some of the men as they resume work which is described as “less
emotionally tiring” and less demanding than caring for a baby. The men recognise the
hard work entailed in caring for a new baby (“No, I would go mad, I’m fairly sure I’d
go mad yeah, I couldn’t be a stay at home dad” Gus) and all express concerns about how
their wife/partner will cope alone. For some there is a palpable sense of relief. As
Gareth says, “it’s very nice to just jump in the van and [...] get back to work,” whilst
for others there is the sudden realisation that managing a job and home life will require
working “harder at both ends.”\cite[370-371]{tri} As James says,


\section{Факторы формирования отцовства}


Далее, разрушаются традиционные мужские роли в семье: <<главы семьи>>,
<<добытчика>>, <<защитника>> и т.д., меняется семейная структура в целом; высок уровень проблем,
связанных с алкогольными, наркотическими и др. зависимости у мужчин и сопутствующим им семейным
насилией и т.д.

От главы семьи к <<добытчику, кормильцу>>. Если в первом качестве отец мог быть оценен только по
критериям
соответствия моральным нормам, то для <<добытчика>> критерием оценки становится его 
профессиональная успешность, движение по карьерной лестнице. 

that fathers experience at various points in their chil-
drenÕs lives (Bozett, 1985; Snarey, 1993). Fathers play
many roles within the family and each of these roles is
associated with a set of ideas, competencies, and ac-
tion patterns. Information exists on the developmen-
tal course for some of the constituent components
(e.g., empathy) related to certain fathering roles (Har-
tup \& VanLieshout, 1995; Turiel, 1997), but almost
nothing is known about how, and under what circum-
stances, these components become integrated and en-
acted over the life course \cite[131]{f21}.


На современном же этапе от мучины требуется не только обеспечивать семью материально, но и уделть 
достаточно времени, внимания и заботы детям. В этом плане выбор количества часов работы, 
продолжительности рабочего дня становится не только вопросом успешности и финансовой независимости, 
но и приобретает моральное звучание. София Бъорк в своем исследовании, посвященном  
продолжительности рабочего дня шведских отцов подчеркивает, что значимым аспектом  выбора 
между частичной или полной занятостью становится необходимость <<показать себя хорошим и 
ответственным отцом в глазах окружающих>> \cite[221]{morfat}.


<<По их мнению, ролевой конфликт мужчины выражается в том,
что требования, предъявляемые к мужской роли, в последние полтора – два десятилетия
стали противоречивыми. От современного мужчины требуется, чтобы он был сильным и
мягким одновременно. Он должен много работать, но при этом ещё и мириться с работой
жены. Помимо всего того, что традиционно должен уметь и делать мужчина,
предполагается его активная включенность в дела семьи. Многим мужчинам не совсем
понятно, в чем же заключается их роль мужа и отца, и им непросто играть прежнюю роль
кормильца семьи, выполняя при этом еще и домашние и материнские обязанности>> \cite[112]{confl}


<<Ролевой конфликт отражается как во внутриличностной, так и в межличностной
сфере. У мужчины появляются ощущения негативного характера: неудовлетворенность,
тревожность, раздражительность, депрессия, снижение самооценки и стресс>> \cite[113]{confl}.



lives (Hofferth, 1999b). Cultural and social changes
have weakened the connection between masculinity
and the expectation of responsible fatherhood (Mar-
siglio, 1998). These changing circumstances have led
to a bifurcation of the adult male population into
those who assume care of their children and those
who do not (Furstenberg, 1988). While some argue
that fatherhood has ceased to be a normative expecta-
tion and has become a voluntary commitment, others
argue that effective fatherhood is an essential quality
of masculinity (Blankenhorn, 1995). Indeed, this de-
\cite[132]{f21}

\subsection{Отцовство и маскулинность}

Отцовство, как было отмечено выше, является одной из сфер самоидентификации мужчины, выступая
важной частью его полоролевой саморепрезентации. Проблемы соотношения маскулинностии отцовства
обретают особую актуальность в связи с динамическим характером обоих явлений. Наблюдаемый в
настоящее время переход от классических к постклассическим моделям маскулинности и отцовства
отражен в исследованиях русскоязычных и зарубежных исследователей. Проблема их соотношения
осмысляется ими в различных вариантах.

В то же время образ маскулинности и отцовства не является неизменным. И.С. Клёцина, анализируя
классические и современные модели маскулинности, предполагает, что современные ожидания от отцов
идут вразрез лишь с классичекой моделью маскулинности, в которую входят следующие положения:
<< 1) необходимость отличаться от женщин; 2)
необходимость добиваться успеха и опережать других мужчин; 3) необходимость быть сильным,
независимым и не показывать слабость; 4) необходимость обладать властью над другими>> \cite{clec}.
В то же время они полностью соответсвуют новой -- гендерной -- теории маскулинности: <<Если в рамках
традиционной модели маскулинности в отношениях с детьми отец демонстрирует эмоциональную
дистанцированность, не включенность в повседневные дела детей, властность, строгость и суровость в
оценке их поступков и поведения, то новый мужчина, напротив, проявляет заботу о детях, устанавливает
с ними доверительные и дружеские отношения, находится в постоянном контакте, чтобы понимать проблемы
и интересы ребенка, и быть объективными, но доброжелательным в оценках поступков детей>>
\cite{clec}.

Так, итальянский исследователь Л.Зойя противопоставляет роли
мужчины и отца, утверждая, что последняя является социальной,
возникшей искусственно на протяжении развития цивилизации. В настоящее время наблюдается 
возвращение мужчин от роли отца к роли самца, основной задачей которого является оплодотворение
самки, а не воспитание и обеспечение выживания потомства. Отцовство формируется благодаря 
образам (архетипам), присутствующим в коллективном бессознательномобщества --- и в 
бессознательном отдельных мужчин. Мужчина, становясь отцом, оказывается на пересечении 
своего биологического и социального измерения, причем его инстинкты не только не упрощают 
выполнение им отцовских функций, но напрямую противоречат им: <<Современный мужчина, даже 
если он намеревается быть отцом, с одной стороны, обладает инстинктом, а с другой --- 
цивилизованностью, которая хранит обряды, необходимые для отца. Фантазия этого мужчины 
обращается к другим женщинам не внутри его цивилизованности, а когда он уходит в отставку 
как отец>> \cite[270]{zo}. Процесс возврата мужчин к биологическим функциям Л.Зойя связывает 
именно с распадом образа отца и отцовства, вследствие которого <<отцовство не имеет больше 
существенного образа. Он зачал физически, но психически не был крещен водой отца>> \cite[270]{zo}.

С другой стороны, изменение ожиданий от отца, изменение его роли --- вместо защитника, добытчика, 
олицетворения власти и главы семьи он больше проводит времени с ребенком и заботится о нем 
приводит к тому, что, во-первых, мужчина оказывается неспособен открыть для ребенка социальное 
измерение жизни, прервать его симбиоз с матерью, а во-вторых, он теряет свою индивидуальность как 
мужчина, становится слишком похожим на мать \cite[285]{zo}. Его образ утрачивает 
яркость в глазах ребенка. Подобный парадокс еще более усиливает противопоставление биологической 
природы мужчины и социальных ожиданий от него: ранее он взял на себя заботу освоем потомстве, что 
не свойственно самцам животных, однако его функция социализации ребенка соответствовала его 
природе; осуществление же ухода за ребенком, игры  с ним, забота о нем --- это женские функции, не 
присущие мужчине.

Парадокс заключается так же и в том, что чаще всего эти функции мужчина выполняет в качестве 
помощника матери ребенка, и выполняет их хуже, чем она. Создается ситуация, при 
которой отец становится словно бы <<второй матерью>>, причем матерью худшей, чем женщина. В таком 
качестве его роль в семье утрачивает свою значимость, и, как следствие --- отпадает 
необходимость и в самом его присутствии.

<<Отеческая специфика заключается именно в этом: он может быть с сыном, когда умеет так же носить 
броню, может бытьотцом, когда он также воин. В отличие от матери,он не может делать только 
что-то одно: Если он будет только носить броню, сын его не узнает; если он не надевает брони 
никогда, его не признают как отца>> \cite[287]{zo}.



Таким образом, на отца возлагаются обширные и часто противоречивые ожидания: с одной стороны, от 
него ожидается большая включенность в процесс ухода за детьми и их воспитания; с другой --- его 
роль в материальном обеспечении семьи все еще остается одним из приоритетов, несмотря на растущую 
значимость женщины в этой сфере. В психологическом смысле отец призывается, по сути, выстроить свой 
собственный симбиоз с ребенком параллельно детско-материнскому, --- и в то же время разорвать его, 
ориентируя себя и ребенка на выполнение социальных функций. Так, важнейшими характеристиками 
вовлеченного отцовтсва называют теплоту, чувствительность, близость, нежность и участие отца в 
занятиях ребенка \cite[129]{f21}

Необходимость финансово обеспечивать ребенка с одной стороны, и уделять ему много времени и 
внимания -- с другой, таким образом, становятся основными компонентами образа ответственного и 
заботливого отца \cite[129]{f21}. Так же 




Фактически нереальный образ идеального отца включается в ценностные приоритеты молодых людей, 
формируя в них отношение к своему будущему отцовству. С одной стороны,
 
Как показывают исследования (\cite{imaf}), подобные коллективные ожидания, связанные с ролью отца 
оказывают влияние на ценностные ориентиры молодых людей, формируя их отношение к будущему 
отцовству. 


% По отношению к этому идеальному отцу: либо развивают гиперответственность и гиперконтроль (как в 
%исследовании --- говоря о готовности к отцовству - надо чтобы и финансово обеспечивал, и чтобы 
%было время на ребенка, и личностная зрелость и чтобы жена тоже была готова к материнству и тоже 
%хотела ребенка и была готова.... ) \cite{imaf}
%Либо вообще не хотят детей, полагая, --- и достаточно справедливо, --- что не смогут удовлетворять 
%тем требованиям, которые общество предъявляет отцам.





\subsection{Стиль отцовства}

Стиль отцовства --- это то, как мужчина реализует свою отцовскую роль: его приоритеты в отцовстве, 
поведенческие реакции, особенности его взаимоотношений с детьми и т.д.

Термин <<стиль отцовства>> используется как в русскоязычной, так и зарубежной литературе. В 
английском языке существуют так же отдельные термины --- fatherhood -- отцовство как 
состояние и fathering -- отцовство как деятельность. Последний соответствует русскоязычному понятию 
<<стиль отцовства>>.

С другой стороны понятие <<стиль отцовства>> тесно связан с понятиями стиля родительства, стиля 
воспитания, стратегии воспитания и т.д. 

Родительское отношение по А.Я. Варге:

Когнитивная составляющая родительского отношения содержит знания и
представления о различных способах и формах взаимодействия с ребенком, об
оптимальной близости в отношениях; знания и представления о целевом аспекте этих
взаимоотношений, а также убеждения в приоритетности тех направлений
взаимодействия с ребенком, которые реализуют родители.
Поведенческая составляющая представляет собой формы и способы поддержания
контакта с ребенком, формы контроля, воспитание взаимоотношениями, путем
определения дистанции общения.
Эмоциональная составляющая включает оценки и суждения о различных типах
родительского отношения, а также доминирующий эмоциональный фон,
сопровождающий поведенческие проявления родительского отношения \cite{varga}

Иными словами, в структруре родительского отношения к детям можно выделить знания и опыт родителей,
их эмоциональное отношение к ребенку (детям) и те способы построения взаимоотношений с детьми,
которые они применяют, исходя из предыдущих двух факторов. Представляется, что эти составляющие,
наряду с перечисленными выше, могут применяться и в отношении детско-отцовских отношений и входить
в понятие стиля отцовства.

В качестве основных факторов, влияющих на выбор стиля родительства исследователи выделяют  
<<социально-культурный уровень
родителей , уровень дохода семьи, уровень
образования каждого из родителей, их ожидания относительно будущей интеграции
ребенка, социальное окружение семьи, степень открытости семьи, а также тип
взаимодействия супругов>> \cite[286]{strat}.

Представляется, однако, необходимым дополнить этот перечень личностными характеристиками мужчин, 
уже являющихся или собирающихся стать отцами. <<Личностные черты родителя
(общительность, гармоничность душевного мира, открытость, способность к
изменениям и анализу, невротичность и другие) являются важной предпосылкой
формирования и проявления родительского отношения. Родительская семья задает
молодому человеку определенную модель, образец будущей семьи или ее анти-идеал.
Общественный уровень влияний задает для мужчин «отправную точку», определенный
образец отцовства, который, преломленный через особенности личности, ценностно-
мотивационную сферу, а также опыт, полученный в родительской семье, дает начало
формированию отцовства в каждом конкретном случае. Совокупное воздействие
факторов приводит к формированию определенной модели отцовства в каждом
конкретном случае \cite[122]{har}


А.Я. Варга выделяет 5 детерминант, которые оказывают влияние на формирование того или иного стиля
родительского отношения к детям: <<особенности личности родителя; клинико-психологические
особенности ребенка; социокультурные и семейные традиции оформления родительского поведения;
этологический фактор раннего контакта ребенка с матерью; особенности общения взрослых членов семьи
между собой>> \cite[16]{varga}

И.С. Клёцина выделяет следующие типы отцов в зависимости от типа маскулинности:

<<Модели отцовского поведения в рамках традиционной модели маскулинности

традиционный отец «старых времен», который заботится о своей семье как руководитель;

«отсутствующий отец» (т. е. отсутствующий, прежде всего в психологическом плане, он может
присутствовать физически, но почти не связан с отцовством);

Модели отцовского поведения в рамках новой модели маскулинности

«ответственный отец» активно включен в процесс ухода за детьми и их воспитания, однако вклад таких
отцов в развитие детей меньше, чем у матерей.

«новый отец» как развивающийся тип мужчины (пеwfаthеr), который не только берет на себя
ответственность за свою семью, но делит поровну с супругой и домашние обязанности, и обязанности по
уходу за детьми, их развитием и воспитанием>> \cite{clec}



Существуют различные типологии стилей родительства. Можно выделить такие их классификации как:

Демократический и авторитарный стили воспитания (Д.М. Болдуин);

Авторитетный, авторитарный, снисходительный и безразличный (Д.Баумринд, Э.Маккоби, Дж. Мартин);

Автократичный, авторитарный, демократичный, эгалитарный, разрешающий, попустительский, игнорирующий
(Д. Элдер) \cite{stil};

Посреднический стиль воспитания: Авторитарный стиль воспитания и Материнский \cite[281]{strat}

Наряду с этими стилями воспитания описаны так же многочисленные нарушения детско-родительских
отношений, такие как гиперопека,  незразрешенные эдипальные конфликты, смешение ролей в семье
(например, сын заботится о опекает мать, играя для нее родительскую роль), симбиотические отношения
и т.д.


Стили воспитания различаются по таким критериям как эмоциональная окраска общения родителей с
ребенком; мера и способ осуществления контроля; распределение семейных ролей между детьми и
взрослыми и т.д. \cite{stil}.

Дэн Аллендер в своей книге <<Как дети воспитывают родителей>> разработал следующее, на наш взгляд,
наиболее общее основание для классификаций родительских стилей: суть детско-родительских
интеракций, согласно Аллендеру, состоит из двух вопросов, которые дети вербально или
через свои действия задают каждому из родителей: <<Любишь ли ты меня?>> (эмоциональный аспект) и
<<Могу ли я делать все что угодно?>> (функциональный аспект) --- и ответов, которые родители
дают на эти вопросы своими словами или действиями \cite{den}. Сочетние этих ответов формирует тот
или иной стиль воспитания: так, родительская позиция <<да, я люблю тебя>> --- <<да, ты можешь делать
все что угодно>> соответсвует попустительскому, снисходительному, а в некоторых случаях
безразличному стилю воспитания; <<да, я люблю тебя>>  --- <<нет, ты не можешь делать все что
угодно>> --- авторитетному, посредническому, разрешающему стилю; <<нет, я не люблю тебя>> --- <<да,
ты можешь делать все, что угодно>>  --- игнорирующему, а  <<нет, я не люблю тебя>> --- <<нет, ты не
можешь делать все, что угодно>> ---авторитарному, автократичному.

Представляется, что подобные основания для классификации, а также сами классификации мы можем
применить не только в отношении родительства в целом, но в частности и отцовства, рассмотрев в этом
ракурсе детско-отцовские отношения.



%Вообще много о стратегиях и стилях воспитания у Орловой.





\subsection{Факторы}


Факторы, влияющие на проявление отцовских чувств:
1. личностные черты характера мужчины,
- в том числе вероисповедание и отношение к нему, система ценностей, физиологические и
психологические особенности
2. система взаимоотношений в семье (у Борисенко – шире – социо-культурный контекст в целом, нормы,
обыча, и, традиции....)\cite{psyot};family background factors (e.g.,
race/ethnicity, parent’s education and economic status, and
family structure and parenting experienced while growing up \cite[164]{long}


Психологическая готовность к отцовству – это внутренняя позиция личности,
стержневой образующей которой является целостная система отношений будущего
родителя к отцовству, которая включает отношение к будущему ребенку, себе как
будущему родителю, к родительской роли, а также к родительству в целом \cite[121]{har}

Таким образом, на формирование представлений об отцовстве влияют:

\begin{enumerate}
	\item Актуальные социокультурные представления об отцовстве, образы <<хорошего>> и <<плохого>>
отца, запечтленные в коллективном бессознательном;

    \item Семейная история мужчины: структура его родительсткой семьи,  модель взаимоотношений
между его родителями, формы реализации отцовстких функций его отцом; возможные травмирующие события
(развод родителей, смерть одного из них, эпизоды сексуального или другогонасилия, перенесенные в
детстве и т.п.).

	\item Структура семьи, созданной мужчиной (наличие/отсутствие оформления брака,
очередность брака, количество, пол и возраст детей, их психологические, физические и
поведенческие особенности, является ли семья многопоколенной (в которой бабушки и дедушки
проживают вместе с детьми или поддерживают с ними постоянный тесный контакт, играя
значительную роль в их жизни), уровень материального достатка семьи, особенности ее социального
положения и т.д.), а так же  особенности коммуникации между ее членами;

	\item Личностные особенности мужчины --- уровень психического функционирования,
уровень интеллектуального и эмоционального развития, тип личности, темперамент, социальная
успешность и т.п.

Именно отражение влияния первых трех групп факторов в сознании мужчины позволяет говорить о них как
о психологических факторах формирования отцовства. Кроме того, именно они определяют степень и
качество влияния на него других вышеперечисленных факторов: идентичные ситуации могут оказывать
различное влияние на различных людей.
\end{enumerate}

Д. Росс изучавший детерминацию отцовства,
выделил факторы, определяющие участие отца
в воспитании ребенка. Это: индивидуальные фак-
торы (отношения, убеждения и мотивация отца; вза-
имоотношения с семьей, в которой вырос; возраст,
когда принял родительскую роль; пол ребенка),
фактор семьи (взаимоотношения матери и ребенка
и взаимоотношения отца и ребенка; взаимоотно-
шения мужа и жены; взаимоотношения отец –
мать – ребенок), внесемейные факторы (взаимоот-
ношения с родственниками; взаимоотношения с со-
седями; взаимоотношения с друзьями), обществен-
ные или формальные факторы (взаимоотношения
с коллегами по работе; система здравоохранения,
роддома), культурные факторы (культура детства
мальчиков; отношение к отцовской роли; убежде-
ния и ценности семьи, связанные с национальны-
ми особенностями) [3, с. 201–221].
Согласно проведенного нами теоретического
анализа структурной модели отцовского отношения,
она включает две подсистемы:
– отношение к себе (представления о себе как
о мужчине, отце; умения, навыки и деятельность
отца, оценка себя);
– отношение к ребенку (представления о ребен-
ке, отношение к ребенку, взаимодействие с ребен-
ком).
Данные подсистемы проявляются в когнитив-
ном, эмоциональном и поведенческом компонен-
тах \cite[40]{otage}




 
\subsection{Реакции на отцовство}

Although we had expected childbirth to be the most
stressful time for men, the studies we examined indicated
pregnancy as being the most stressful period (30,33–35).
This finding could be because men must undergo psycho-
logical reorganization during this time (33) when they
may also be aiming to adhere to a father image based
on close involvement with family and childcare—an
image for which they never inherited a role model (36).
 \cite[313]{flit}.


 Seven studies documented fathers’ experiences dur-
ing labor and delivery, finding that fathers frequently
felt helpless, useless, and anxious during the labor pro-
cess (21,24,29,30,38,39). In fact, fathers usually reported
that they had not expected labor to be so demand-
ing (21,30), and had felt mostly out of place, vul-
nerable, unprepared, and in need of psychological
support (26,39,40)\cite[313]{flit}


process (23,25,26). The main difficulties they encounter
during this time are linked to diminished indepen-
dence. Moreover, they are forced by the events to
learn a new approach to life, so as to deal with a
new sense of feeling powerless—of not having control
over the situation—and to accept that they are enter-
ing a new phase of their lives (25,27). The conflicting
mental states produced by these components are fre-
quently expressed through moodiness, irritability, anx-
iety (33), frustration (26,27), and a negative perception
of self (23). Participation in labor and delivery is an
important factor for men in learning to cope with this
transition (39), because the process is accompanied not
only by the previously cited feelings of helplessness
and other factors but also by a sense of pleasure and
pride (21,30,39) \cite[314]{flit}


After birth, new fathers faced the difficult task of
putting their own fatherhood image (about which they
had fantasized during the pregnancy period) into prac-
tice (31,32,34,36). Fathers who managed to construct a
positive, relatively conflict-free father self-image expe-
rienced a motivational force for greater involvement
with their infants (32). If their self-image, however, still
involved many conflicting elements, they found it dif-
ficult to deal with the demands of their new role and
easily experienced feelings of exclusion (29,32).\cite[314]{flit}

Даже те мужчины, которые заинтересованы в активном участии в воспитании ребенка и уходе за ним,
испытывают сложности во время беременности партнерши, поскольку это сталкивается с их
представлениями о равных гендерных ролях . However, the pregnancy period was
particularly problematic for these
men as embodied experiences marked a dramatic gender separation, which contrasted
with their otherwise equitable roles (Doucet, 2007). In the first interview, Rick artic-
\cite[20]{long}.

так же мужчины чувуствут себя исключенными из-за грудного вскармливания \cite[21]{long}

Сначала думают, что будут больше вовлечены, когда дети подрастут, и не будут физически зависеть от
матери, а потом - ждут когда дети совсем вырастут ине будут так много требовать от них. При этом
отцы становятся более вовлеченными в воспитание, общение и уход за детьми, но не чувствуют
удовлетворения и остаются ориентированными на будущее \cite[22]{long}.


Предположения до рождения детей
This centres on their changing per-
ceptions of a self now imagined as an involved father—being mature, having new
responsibilities and interacting with the baby/child—and how this will fit into other
areas of their lives and most significantly, paid work. Interwoven across the narratives
are recognisable, gendered patterns and associations between men’s lives and paid
work, responsibilities and economic provision: but also more optimistic and different
ways of “being there” and sharing caring for their child. Most of the men spoke of
thinking that at some time in their life they would become a father. Confirmation of this
event is greeted as having “kind of proved your virility” and it being “part of the life
plan” leading some to talk of feeling “quite grown up” and others of the need to “grow
up” and “life becoming more serious.”\cite[367]{tri}

For others fathering involvement is something that, following paternity leave, will
take place during “evenings,” “nights,” “weekends” and “holidays.” It will involve “as-
sisting,” “supporting” and “helping” their wife/partner in “lots of tasks.” These include:
“feeding,” “changing nappies,” “bathing,” “dressing,” “putting it to bed” and “getting
up in the night and seeing what’s wrong.” Whilst the men talk about sharing care and
“home life balance,” in reality their availability will be significantly shaped by their
(continued) workplace commitments. Fathering in practise will be for most, “task” and
“activity” focused and something to be “fitted in.” \cite[369]{tri}As Dylan says:


.

Вообще на мужчинах по-разному отражается отцовство. Содной стороны, это важная ценность, одна из
значимых сфер самоидентификации - а с другой из-за полоролевого конфликта  Мужчина попадает в
ситуацию ролевого стресса, так как не может
качественно выполнять требования новой ситуации, вследствие чего происходит
ослабление власти в семье, появление субъективного ощущения отцовской
некомпетентности, следствием которого является появление у мужчины-отца
субъективных ощущений негативного характера: неудовлетворенности, тревоги,
раздражительности, обидчивости \cite[111]{confl}.






\section{Социокультурная динамика отцовства}



Годфруа Ж 
И.С. Кон Этнография родительства
ИС Кон -
,
В.Ф. Анурин, мерещакова verlinden m
т.в. архирева
р.в. овчарова
laRossa r reitzes d  - о социокультурной стороне О.
Все говорят о том, что сейчас очень высоки требования к вовлечености отца в жизнь детей. Уровень ее 
фактически приблежается к материнскому. То есть от функций защитника, кормильца, носителя власти, 
примера для подражания – к болле личностному восприятию.

Крайг г – о в разных культурах.



с.79 – структура отцовства
найти
2 подхода к отцовству – 1. исходяот ребенка,
2. от личности родителя \cite[11]{psyot}, нам надо второй
с 11, мы в рамках данного подхода будем говорить о том, какие факторы влияют на реализацию 
отцовства.
Орисенко, с.35 – внешние и внуиренние факторы ф ыормирования личности. На формирование отцовства так 
же влияют внутренние и внешние факторы \cite[35]{psyot}.
3. его взаимоотношения с его отцом \cite[49]{rah}


становление отцом активизирует влияние всех этих факторов.

методики там же




\printbibliography[env=gostbibliography,sorting=ntvy]
\addcontentsline{toc}{chapter}{Литература}

\end{document}

\documentclass{../../common/thesisbyxetex}

\usepackage{enumitem}
\setlist{nolistsep}	% отступы между элементами перечесления

\addbibresource{../../common/bibliography.bib}

\begin{document}

\hypersetup{
pdftitle = {Контрольная работа, Общая психология},
pdfauthor = {Станкевич Наталия Александровна},
pdfsubject = {контрольная},
pdfkeywords = {эмоции, психология, контрольная}
}% End of hypersetup

\begin{titlepage}
\newpage

\begin{center}
\large \uppercase{Белорусский государственный университет \\
факультет философии и социальных наук\\
кафедра психологии}
\end{center}
 
\vspace{12em}



\begin{center}
\Large \uppercase{\textbf{Функции эмоций}}
\end{center}

\begin{center}
\textbf{Контрольная работа по дисциплине <<Общая психология>>}
\end{center}

\vspace{11em}
 
\begin{flushright}
%\parbox{0.45\textwidth}{
Выполнила:

\vspace{0.25em}

студентка 1 курса отделения психологии

\textbf{Станкевич Наталия Александровна}

\vspace{2em}

Проверила:

\vspace{0.25em}

\textbf{Довнар Анастасия Евгеньевна}

%}%end parbox
\end{flushright}

\vspace{2em}
Дата сдачи на кафедру:\underline{~~~~~~~~~~~~~~~~~}
\vspace{0.25em}

Методист:
 
\vspace{\fill}

\begin{center}
Минск, 2015
\end{center}

\end{titlepage} 

\tableofcontents 
\section{Понятие и функции эмоций}
Для полного понимания функциональных особенностей эмоций, необходимо отметить их 
двойственный характер. Эмоции находятся словно на пересечении различных сфер, для их описания 
необходимо изучение следующих дихотомий:

 \subsection*{Биологическое --- психологическое}
  
Рассуждая о природе эмоций, ученые выделяют как связь эмоций  с биологическими процессами 
(учащением или замедлением серцебиения, изменениями давления, температуры тела, мимики и 
пантомимики, а также различные психосоматические проявления и заболевания), так и с 
психологическими. Примером последних может, например, служить концепция эмоциональных пусковых 
механизмов, принятая, в том числе, в психоанализе. Суть ее состоит в том, что иногда эмоциональная 
реакция человека на некий стимул явно завышена, то есть на незначительное воздействие 
(слово, фразу или ситуацию) следует мощная реакция (например, в форме гнева или чувства вины или 
стыда). В этих случаях говорят. что стимул стал пусковым механизмом эмоции. Объяснение этому 
явлению 
кроется, как правило, в прошлом (детстве) человека, когда подобный стимул был сопряжен для него с 
сильными --- и как правило болезненными --- переживаниями. 

Дихотомия <<биологическое --- 
психологическое>> нашла свое отражение в определении эмоции, которое предлагает К.Изгард в своей 
теории дифференциальных эмоций: <<Эмоция --- это сложный процесс, имеющий нейрофизио-логический, 
нервно-мышечный и феноменологический аспекты. На нейрофизиологическом уровне эмоция определяется по 
электрохимической активности нервной системы, в частности, коры, гипоталамуса, базальных ганглиев, 
лимбической системы, лицевого и тройничного нервов. На нервномышечном уровне эмоция --- это прежде 
всего мимическая деятельность, а вторично — пантомимические, висцерально - эндокринные и иногда 
голосовые реакции. На феноменологическом уровне эмоция проявляется либо как сильно мотивированное 
переживание, либо как переживание которое имеет непосредственную значимость для субъекта>> 
\cite[65]{tde}

Подобный же подход характерен и для  исследований эмоций, проходивших во второй половине ХХ века, 
как русскоязычных, так и зарубежных: психология в их трудах практически неотделима от физиологии, и 
потому функции эмоций рассматриваются ими либо с позиций их физиологических проявлений, либоих 
важжности для физического выживания индивидуального организма или популяции в целом 
\cite{sem,ilemo,dode, gel}.


\subsection*{Психологическое --- социальное}

Эмоции, очевидно, могут быть значимыми как для самого индивида, его внутреннего 
психологического пространства, так и для построения (или разрушения) его социальных связей в 
обществе. Так, патологическая тревога может отрицательно влиять на качество жизни 
человека и даже провоцировать у него соматические нарушения --- и в то же время сказываться на его 
общении с другими (например, порождая излишнюю застенчивость и тем самым осложняя коммуникацию).

\subsection* {Положительное --- отрицательное}

Эти характеристики применяются как в отношении самих эмоций, так и их функций. Например, радость, 
интерес, восхищение, нежность традиционно относятся к положительным эмоциям, так как их приятно 
ощущать, переживать. По той же причине, соответственно, стыд, вина, страх, зависть будут относиться 
к эмоциям отрицательным. По функциям эмоции могут быть как созидательными, так и разрушительными.
Следует, однако, отметить, что данное противопоставление не столь однозначно. Более верно было бы 
говорить о том, что каждая эмоция оправданна в соответсвующих ситуациях, и что каждая из них может 
какс пособствовать построению здоровых отношений между людьми, так и разрушать эти отношения. 
Например, гнев может способствовать ссорам, тем самым затрудняя общение --- но вместе с тем 
энергия, которую он дает, необходима для отстаивания собственных границ и является нормальной 
реакцией на ситуации несправедливости.

Е.~П.~Ильин разделяет понятия роли и функции эмоций, соотнося функцию эмоций с биологической 
сферой, а роли эмоций - с социальной: <<Функция эмоций — эхо узкое природное предназначение, 
работа, 
выполняемая эмоциями в организме, а их роль (обобщенное значение) - это характер и степень участия 
эмоций в чем-либо, определяемая их функциями, или же их влияние на что-то помимо их природного 
предназначения>> \cite[100]{ilemo}. 
Поскольку, согласно Ильину, функции эмоций относятся к биологической сфере, то они не могут быть 
отрицательными: если они сохраняются в процессе эволюции, --- значит, они необходимы как людям, так 
и животным. Что касается ролей, которые выполняют эмоции, то они, как и сами эмоции, могут быть как 
положительными, так и отрицательными. При этом отрицательные эмоции (гнев, страх, тревога) играют 
более важную роль в развитии личности: они стимулируют человека на поиск состояния равновесия с 
системой. в которой он существует, в то время как эмоции положительные стимулируют его для 
дальнейшего развития отношений с системой. И те. и другие эмоции связаны с другими сферами 
человеческой жизни: мотивацией, потребностями. эмоциональной регуляцией и т.д.

Исследователь выделяет следующие роли эмоций:
\begin{enumerate}
 \item Мотивационная
 \item Эмоции как оценка значимости внешнего раздражителя --- благодаря эмоциям происходит 
распознавание того или иного внешнего раздраителя как важного или неважного, приятного или 
неприятного, опасного или безопасного - как в с биологической, так и с личностной точек зрения. 
 \item Эмоции как сигнал о появившейся потребности --- эмоции составляют основу психологического 
механизма формирования потребностей.
 \item Эмоции как способ маркировки значимых целей --- благодаря эмоциям цели, которые человек 
ставит перед собой, становятся для него значимыми не только утилитарно, но и психологически.
 \item Эмоции как механизм, помогающий принятию решения
 \item Побудительная роль эмоций
 \item Роль эмоций в оценке достигнутых результатов
 \item Эмоция как ценность и потребность
 \item Потребность в эмоциональном насыщении
 \item Активационно-энергетическая роль. 
 \item Коммуникативная роль эмоций. 
 Эмоции составляют невербальный пласт любой коммуникации. Мимика, пантомимика, тембр и громкость 
голоса, интонации, меняющиеся в зависимости от эмоционального состояния говорящего несут 
значительный объем транслируемой информации и оказывают сильное влияние на построение диалога.
 \item Оздоровительная роль эмоций.
 Речь идет о соматических проявлениях эмоций, которые могут иметь как разрушительный. так и 
оздоровляющий характер, например, провоцируя возникновения различных заболеваний или, наоборот,  
улучшая физическое состояние человека \cite[110-117]{ilemo}.
\end{enumerate}   

При этом роль эмоций не всегда положительная. Отрицательная роль эмоций на биологическом уровне 
проявляется в ухудшении физического состояния человека вследствие постоянно или длительно 
испытываемых им негативных эмоций: страха, гнева, стыда и т.д. 
На социальном уровне эмоции могуть выступать средством манипулирования 
другими людьми  \cite[117]{ilemo}.
 
Подобный подход к систематизации функций эмоций изложен в труде П.В. Симонова <<Эмоциональный 
мозг>>, в котором автор выделяет следующие функции эмоций:
  \begin{itemize}
    \item Отражательно-оценочную, суть которой сводится к тому, что <<эмоции выступают в роли 
своеобразной «валюты мозга» — универсальной меры ценностей>> \cite[27]{sem}. Данная функция 
становится, таким образом, как бы соединительным звеном между человеком и внешним миром, делая 
происходящее в нем личностно значимым для человека. В то же время эмоции формируют отношение 
человека к происходящему- положительное или отрицательное, включая таким образом полученный опыт в 
его ценностную иерархию.
    \item в соответствии с полученным опытом человек регулирует свое поведение, и потому эмоции, 
согласно П.В. Симонову, имеют также ряд регуляторных функций, таких как 
переключающая, подкрепляющая, компенсаторная и т.д \cite{sem}.
\end{itemize}
  

Вышеперечисленные функции можно назвать общими для всех эмоций. В то же время можно говорить и о 
том. что различные эмоции выполняют и другие, специфические, функции.
Например, известный исследователь эмоций К.Изард, автор дифференциальной теории эмоций, выделяет 
следующие наиболее важные эмоции и их функции в биологическом и социальном пространстве 
существования индивида.

 \subsection*{Интерес}
 
 К биологической функции интереса относится, помимо прочего, распределение энергии, получаемой 
организмом, на концентрацию внимания на конкретном объекте или задаче, что помогает выполнить ее 
более качественно. Одним из биологических выражений интереса и концентрации внимания является 
брадикардия, благодаря которой достигается состояние покоя и сосредоточенности \cite[126]{izpsy}.
К социальным фунциям интереса относятся следующие:
\begin{itemize}
 \item Мотивационная. Интерес связан с внутренней мотивацией человека, благодаря которой его 
энергия устремляется на решение той или иной задачи. Интерес может быть силен настолько, что 
превосходит даже физиологические потребности: увлекшись чем-то, мы можем забывать о потребности в 
сне, еде и т.д. Благодаря интересу значиттельно повышаются способности человека к научению, 
концентрации внимания, запоминанию информации.
\item Социализирующая функция интереса. Благодаря социальному интересу у человека формируются 
навыки игровой деятельности и социальной коммуникации. Игра, особенно в детском возрасте, имеет 
огромное значение для когнитивного и эмоционального развития индивида. Тренируя исполнение 
различных социальных ролей, а также разряжая отрицательные эмоции в ходе игры, ребенок 
совершенствуется в социальной адаптации \cite{izpsy}.

\end{itemize}

\subsection*{Радость}

К.Изгард относит радость, как и интерес, к базовым эмоциям. В социальном плане радость имеет 
первостепенное значение в процессе формирования привязанности между младенцем и матерью, а она, в 
свою очередь, необходима для установления базового доверия ребенка  к себе, окружающим и миру в 
целом. У взрослых радость способствует развитию социального общения, доверия к 
окружающим \cite[154]{izpsy}.

В биологическом смысле радость оказывает благотворное воздействие на регенерационные процессы в 
организме, в том числе ускоряя выздоровление после перенесенных заболеваний, восполняет потраченную 
энергию.

\subsection* { Печаль}

Изард пишет: <<Эмоция печали выполняет ряд психологических функций. Переживание горя сплачивает 
людей, укрепляет дружеские и семейные связи; печаль тормозит умственную и физическую активность 
человека, и тем самым дает ему возможность обдумать трудную ситуацию; она сообщает человеку и 
окружающим его людям о неблагополучии, и наконец, печаль побуждает человека к восстановлению и 
укреплению связей с людьми>> \cite[211]{izpsy}.

Реакция острого горя, связанного обычно с потерей близкого человека, не только способствует 
сплочению людей и укреплению социальных связей, но и дает возможность свыкнуться с утратой и 
продолжать жить.

\subsection*{ Гнев}
Гнев считается отрицательной эмоцией, поскольку его проявления могут нарушать отношения между 
людьми. Люди часто со стыдом и чувством вины вспоминают о происходивших с ними вспышках гнева. 
которые провоцировали ссоры и напряженные отношения с окружающими. Вместе с тем, как отмечает 
К.Изард, гнев является нормальной реакцией. Например, на ситуации ограничения свободы, на 
препятствия 
на пути к цели \cite{izpsy}. Добавим. что в более широком смысле гнев является универсальной 
реакцией на ситуации несправедливости, при этом не имеет большого значения, относится ли 
несправедливость к нам или третьим лицам. В качестве таковой, гнев дает человеку необходимую 
энергию, чтобы отстаивать собственные границы: <<Современный человек достаточно редко оказывается в 
ситуации физической угрозы, но довольно часто ему приходится защищать себя психологически, и в этих 
случаях умеренный, регулируемый гнев, мобилизуя энергию человека, помогает ему отстаивать свои 
права. Если кто-то угрожает вашей психологической целостности, с ним следует обращаться твердо и 
решительно, и основой этой решительности может быть умеренное чувство гнева. Ваше негодование 
принесет пользу не только вам, но и тому человеку, который, нарушая закон или установленные 
обществом правила поведения, подвергает опасности вашу жизнь и жизнь других людей>> 
\cite[254]{izpsy}

\subsection*{ Страх и тревога}

Страх часто относят к негативным эмоциям, поскольку переживание его ощущается человеком как весьма 
неприятное. Сильный страх, паника приводит к утрате человеком контроля над собой и своим телом: 
нередки случаи, когда страх парализует или, наоборот, заставляет обратиться в безудержное бегство. 
Вместе с тем, контролируемое чувство страха сигнализирует об опасности, давая возможность 
подготовиться к опасной ситуации, мобилизироваться. Кроме того, в социальном плане страх может 
становиться причиной для объединения людей. Изард приводит пример о соседях, одного из которых 
ограбили. Остальные жильцы объединяются с тем, чтобы предотвратить другие грабежи \cite{izpsy}. В 
то же время сильный экзистенциальный страх, например, страх смерти, провоцирует так называемые 
пограничные ситуации, которые, напротив, противопоставляют людей друг другу.

\subsection*{ Стыд}

Функции стыда достаточно противоречивы. Изгард выделяет следующие из них: <<Средоточие эмоции 
стыда находится в «Я» или в некоторых аспектах «Я»; стыд активирует самооценку.
Обостренный самоотчет и преувеличенное самоосознание, вызванные стыдом, пробуждают все более 
отчетливые образы «Я». Осознание, сопутствующее переживанию стыда, способствует усилению «Я», 
уменьшению уязвимости личности>> \cite[355]{izpsy}.
В то же время стыд может делать человека и более уязвимым: <<cтыд разоблачает «Я». Человек, 
переживающий стыд, более уязвим>>\cite[355]{izpsy}.
Далее, <<переживание стыда возникает в ответ на высказывания и поступки окружающих людей, и 
этот факт обеспечивает известную степень сенситив-ности в отношении мнений и чувств других людей, 
особенно тех, к которым мы эмоционально привязаны и чьим мнением дорожим>> \cite[355]{izpsy}.
Кроме того, 	<<Обостренный самоотчет и стыдливый румянец, вызываемые переживанием стыда, 
пробуждает более  острое, нежели другие эмоции, осознание собственного тела>> \cite[355]{izpsy}.
 
Вместе с тем, стыд <<крайне враждебен по отношению к рациональным, интеллектуальным процессам. 
...По сравнению с эмоцией вины, эмоция стыда — менее дифференцированная, более иррациональная, 
более 
примитивная, труднее вербализуемая реакция, содержание которой почти не поддается осмыслению>> 
\cite[355]{izpsy}

 Наконец, <<несмотря на то что человек в стремлении избежать неприятных ощущений, связанных со 
стыдом, может поддаться искушению конформизма, нельзя не отметить, что противостояние стыду и 
успешное преодоление переживания стыда способствуют развитию личностной автономии и идентичности, 
способности к зрелым чувствам>> \cite[355]{izpsy}. 

\subsection* { Вина}

Изгард описывает функции вины следующим образом: <<Ожидание вины становится основой личностной 
ответственности. Эмоция вины совместно с эмоцией стыда лежит в основе чувства социальной 
ответственности и становится расплатой за проступки.
Специфическая функция эмоции вины заключается в том, что она стимулирует человека исправить 
ситуацию, восстановить нормальный ход вещей. Если вы чувствуете себя виноватым, у вас возникает 
желание загладить свою вину или хотя бы принести извинения человеку, перед которым вы провинились. 
Такое поведение — единственно эффективный способ разрешения внутреннего конфликта, порожденного 
виной>> \cite[380]{izpsy}.

Своеобразный подход к пониманию функциональной сущности эмоций высказывает исследователь Б.И. 
Додонов в своей книге <<Эмоция как ценность>>. Описывая  эмоции как одну из составляющих 
аксиологического пространства человека, ученый выделяет, прежде всего, такие их функции как 
мотивационная (эмоциональный заряд дает необходимую энергию для выполнения какого-либо действия), 
ориентационная (окрашивая события и явления в негативные или позитивныетона, эмоции служат 
своеобразным <<навигатором>> для человека, помогая ему принимать решения, исходя из его 
ценностныхориентаций), функция обобщения (подобные события или явления окрашиваются одинаковыми 
эмоциями) и т.д. \cite{dode}.


Современные исследования выводят на первый план социальное значение эмоций, их роль в формировании 
качества жизни человека и его коммуникации с другими людьми. Акцент делается на причинах, 
побуждащих нас испытывать те или иные эмоции, на триггерах, формирующих наши эмоциональные 
состояния. Мимические и пантомимические выражения той или иной эмоции рассматриваются в контексте 
невербального общения, их изучение и описание призвано помочь собеседникам более точно понимать как 
другого, так и самого себя. Появляется большое количество популярной психологической литературы, 
призванной научить читателей понимать <<язык жестов>>, <<язык тела>> и т.д.

Одновременно нивелируется деление эмоций на положительные и отрицательные. Признается важность 
любых эмоций для гармоничного развития личности. Что касается функций эмоций, то их исследование 
смещается из поля нейрофизиологии в поле социального. Исследователи эмоций больше задаются 
вопросами не о том, какие биологические механизмы задействованы в появлении эмоций, а о 
психологических триггерах, которые определяют силу и направленность эмоциональных реакций, а также 
о том, как лучше понимать свои реакции и управлять ими. Также изучаются функции конкретных 
отдельных эмоций.

\section{Роль эмоционального воспитания в гармоничном развитии ребенка}

Среди стратегий воспитания детей в последнее время набирают популярность программы раннего развития 
малышей. Уже детям 2-3 лет предлагаются игровые занятия математикой, чтением и иностранными 
языками. Такой уклон в интеллектуальное развитие ребенка, однако, часто приводит к тому, что из 
поля зрения родителей и воспитателей выпадают вопросы, связанные с развитием эмоциональной сферы 
ребенка. В результате растет количество жалоб взрослых на <<неуправляемых>>, <<истеричных>> детей, 
учащаются случаи неврологических расстройств у детей, в частности СДВГ.

Вместе с тем исследования многих знаменитых психотерапевтов говорят о том, что эмоциональное 
воспитание ребенка, внимание к его внутреннему миру имеет гораздо большее значение, чем раннее 
умения считать или читать. Более того --- формирование этих способностей и навыков напрямую 
связано с развитием воображения, образного мышления, памяти, умением мыслить абстрактно --- то 
есть со сферой чувств, потребностей и склонностей ребенка. Например, педагог и математик А.К. 
Звонкин в своей книге-дневнике <<Малыши и математика>>, в которой он описывает свои занятия 
математикой с 3-4-летними детьми, приходит к выводу о том, что обучение, построенное на объяснении 
материала, не дает понимания детьми предмета. Прогресс происходит только тогда, когда ребенок готов 
(когнитивно, эмоционально) к какому-либо <<открытию>>  --- будь то геометрические формы, закон 
сохрания объема вещества или что-либо еще \cite{zvon}.  

Важность эмоциональной составляющей в развитии ребенка подчеркивали такие исследователи как 
Маргарет Малер, Джон Боулби, Дональд Вудс Винникотт, Франсуаза Дольто, В.А. Ильин и др.
М.Малер в своем исследовании <<Психологическое рождение человеческого младенца: симбиоз и 
индивидуация>> подчеркивает значимость психологического комфорта ребенка в его отношениям с 
матерью. Теплое, спокойное, доброжелательное и полное любви присутствие матери с младенцем 
позволяет тому сформировать устойчивое позитивное отноешние к ней, к самому себе и к миру: 
<<мама --- хорошая, я --- хороший, --- мир хороший>>. Это отношение, соответствующее эриксоновской 
стадии базового доверия, позволяет малышу в дальнейшем без проблем перейти к следующей фазе своего 
развития --- отделению от матери. <<Хорошая мама>>, будучи интроецирована им, даже в моменты 
физического отсутствия рядом, продолжает эмоционально согревать и поддерживать его, давая силы и 
энергию для изучения окружающего мира \cite{maler}. 

В то же время мать, которая часто оставляет 
младенца и/или эмоционально отвергает его, холодна с ним, формирует в ребенке повышенную 
тревожность по отношению к ней, к миру и к самому себе, формируя, в терминах Эриксона, базовое 
недоверие к окружающему. В результате мамы жалуются на то, что ребенок <<всего боится>>, <<с рук 
не слезает>>, не может ни минуты побыть один, не умеет общаться с другими детьми и т.д. 
Психосоматическими проявлениями повышенной тревожности становятся также заикание, всевозможные 
тики,энурез.

Те же процессы, но в несколько иной терминологии --- с точки зрения формирования привязанности 
--- раскрывает Дж.Боулби. Нарушение эмоционального контакта с матерью приводит к деформации 
привязанности по двум типам. Первый вариант --- тревожная привязанность --- характеризуется 
поведением, описанным выше. Можно только добавить, что способом привлечения внимания 
<<ускользающей>> матери могут быть различные болезни: если ребенок получает материнскую заботу и 
внимание только будучи больным --- его организм может начать <<выдавать желаемый результат>>, 
реагируя различными нарушениями. 

Обратный вариант нарушения привязанности --- <<избегающая>>, когда ребенок замыкается в  себе, 
теряя способность просить и принимать помощь и заботу даже в тех ситуациях, когда они ему 
необходимы \cite{boulby}.

Первостепенную важность эмоционального контакта ребенка с окружающими показывают так же 
исследования Р. Шпица, посвященные проблемам эмоциональной депривации младенцев. Описывая такие 
серьезные психосоматические нарушения у детей, как госпитальный синдром и анаклитическая депрессия, 
он пришел к выводу о том, что недостаток эмоционального общения настолько же губителен для младенца 
как и физическая депривация \cite{spitz}. О том же говорит и эксперимент <<каменное лицо>>, в 
котором матери младенца предписывалось не реагировать мимически на изменения выражения лица 
ребенка, сохраняя невыразительное, безэмоциональное выражение лица, что провоцировало сильнейший 
стресс у малыша.

Вместе с тем, думается, многие молодые родители сегодня не придают должного внимания развитию 
эмоциональной сферы ребенка. Ситуация усугубляется тем, что зачастую даже при наличии желания, у 
них отсутсвуют необходимые знания о том, как помочь этому развитию. Проблема заключается и в 
том,что сами взрослые  в силу различных причин испытывают сложности с распознаванием, называнием и 
тем более управлением своими эмоциями. Между тем, именно умение точно распознать и назвать свои 
эмоции является базисным для гармоничного развития личности. Ребенок, не умеющий определить свои 
чувства, не понимающий самого себя, сталкиваясь с эмоциями и аффектами, будет <<затоплен>> ими, не 
сможетих контролировать. Необходимо учитывать, что эмоциональные переживания ребенка гораздо 
сильнее, чем у взрослого --- а психика его слабее, поскольку он не обладает его жизненным опытом. 
Неконтролируемые эмоции становятся двигателем, в частности, детских истерик, начинающихся, 
как кажется, без всякой внешней причины. На самом деле именно с помощью истерики ребенок сбрасывает 
нервное напряжение, которое, умей он обращаться со своими эмоциями, он мог бы разрядить гораздо 
более экологичными способами.

Узнавать свои эмоции ребенок учится от матери (и \ или лиц, которые заботятся о нем). Обычно, если 
мать эмоционально открыта на ребенка, это обучение происходит спонтанно. Уже с первых дней жизни 
мать улавливает выражение личика младенца и мимически копирует его - отзеркаливает. При этом может 
добавляться называние мимики и  эмоции: <<о, ты улыбаешься ---обрадовался>>,  <<наморщился  --- 
сейчас заплачешь>>, <<сердишься>>, <<ой, как страшно>> и т.д. Поощряемый таким образом, ребенок 
учится соотносить свои переживания с конкретными мимическими выражениями и словами (очевидно, что с 
возрастом его <<эмоциональный словарь>> будет пополняться, а градация эмоций становиться все более 
тонкой).

Препятствиеми для этого процесса может стать, например, эмоционально холодная, закрытая мать, чье 
общение с ребенком сводится только к физическому уходу за ним. Далее, мать может, как уже 
говорилось, сама не распознавать своих эмоций (алекситимия) и, соответственно, не иметь возможности 
адекватно считывать и называть эмоции ребенка. 

Однако эмоции ребенка должны быть не только названы матерью, но и приняты ею. Принятие означает, 
что, во-первых, мать выдерживает эмоциональный накал ребенка, (а не, например, впадает в ярость или 
не пугается в ответ на его агрессию), а во-вторых --- мать не разделяет чувства на <<хорошие>> и 
<<плохие>>, <<дозволенные>> и <<запретные>>. Последнее, разумеется, тесно связано с идеальным <<Я>> 
матери и ее способностью принимать самые разнообразные чувства в самой себе. Именно поэтому 
для ребенка более здорово иметь <<достаточно хорошую мать>> (в терминологии Д.В. Винникотта), чем 
мать идеальную. Достаточно хорошая мать чутко реагирует на потребности, в том числе и 
эмоциональные, своего ребенка, и адекватно их удовлетворяет, она эмоционально вовлечена в процесс 
общения с ним. При этом она может ошибаться, испытывать разные, в том числе отрицательные эмоции и 
состояния (гнев, скуку, раздражение, усталость и т.д.). Однако она принимает в себе слабости, и 
тем 
самым учит этому ребенка \cite{vindeti,vinraz}. В этой ситуации эмоциональный мир ребенка 
развивается гармонично, давая надежную опору для развития его в других сферах, и в полной мере 
реализуя все функции, о которых шла речь выше.








\printbibliography[env=gostbibliography,sorting=ntvy]
\addcontentsline{toc}{chapter}{Литература}

\end{document}

\documentclass{../../common/thesisbyxetex}

\geometry{xetex,a4paper,top=2cm,bottom=2cm,left=2cm,right=2cm}


\usepackage{enumitem}
\setlist{nolistsep}	% отступы между элементами перечесления


\addbibresource{../../common/bibliography.bib}

\begin{document}



\begin{titlepage}
\newpage

\begin{center}
\large \uppercase{Белорусский государственный университет \\
факультет философии и социальных наук\\
кафедра психологии}
\end{center}
 
\vspace{12em}

\renewcommand{\mkgostheading}[1]{#1}

\begin{center}
\Large \uppercase{\textbf{Анализ статистического исследования}}
\end{center}

\begin{center}
\textbf{Контрольная работа по дисциплине <<Статистические методы в психологии>>}
\end{center}

\vspace{11em}
 
\begin{flushright}
%\parbox{0.45\textwidth}{
Выполнила:

\vspace{0.25em}

студентка 1 курса отделения психологии

\textbf{Станкевич Наталия Александровна}

\vspace{2em}

Проверила:

\vspace{0.25em}

\textbf{Фабрикант Маргарита Сауловна}

%}%end parbox
\end{flushright}

\vspace{2em}
Дата сдачи на кафедру:\underline{~~~~~~~~~~~~~~~~~}
\vspace{0.25em}

Методист:
 
\vspace{\fill}

\begin{center}
Минск, 2015
\end{center}

\end{titlepage}

Данная работа выполнена на основе статьи  Р. И. Остапенко <<Самодиагностика как условие 
формирования математической компетентности студентов психологических специальностей>>\cite{ostap}.

Статья направлена на описание использования практико-ориентированного, увлекательного материала в 
ходе занятий по предмету <<Математические методы в психологии>> для студентов психологических 
специальностей. В данном случае материалом для исследования являются результаты самодиагностики 
студентов на предмет их ситуативной и личностной тревожности, а также степени стресса, выполненной 
с помощью специализированных методик: теста «Исследование тревожности (опросник Спилбергера) и 
методики определения стрессоустойчивости и социальной адаптации Холмса и Раге. Перед студентами 
стояла задача  определить наличие зависимости между этими факторами, используя статистические 
методы обработки данных (с помощью программы SPSS).  

Выборка состояла из 15 человек. В данной работе используются, соответственно, 3 \textbf{переменные}:
\begin{itemize}
 \item Ситуативная тревожность,
 \item Личностная тревожность,
 \item Степень стресса.
\end{itemize}

Рабочая \textbf{гипотеза} утверждала наличие прямой зависимости между степенью стресса и 
выраженностью ситуативной и личностной тревоги: было сделано предположение о том, что чем более 
человек подвержен стрессу, тем более высокий уровень тревожности будет у него наблюдаться.

Для проверки данной гипотезы были  применены следующие \textbf{методы}:

\begin{enumerate}

 \item После составления матрицы исходных данных столбцы данных были попарно обработаны с помощью 
коэффициента корреляции Пирсона с целью получения корреляционной матрицы (корреляционный анализ). 
Результаты были таблично представлены матрицей корреляции, а также графически отображены в 
графической матрице и корреляционном ненаправленном графе.

\item Была построена регрессионной модели (множественный регрессионный анализ), где в 
качестве независимых переменных выступают показатели тревожности, а зависимой – степень стресса.
\item На основании коэффициентов регрессии и ковариации было установлено, что связь между 
личной и ситуативной тревожностью статистически значима, также значимо влияние ситуативной 
тревожности на степень стресса.
\item Были выдвинуты три производных гипотезы о распределении причино-следственных связей 
между переменными, для каждой из которых была проведена проверка на статистическую состоятельность.

\end{enumerate}

Представляется, что описанные в статье методы адекватны исследуемой задаче и полностью ее 
исчерпывают. В этой связи предложенные ниже дополнительные методы могут служить только для 
углубления исследования и уточнения гипотезы. В качестве таких методов могут выступить:
\begin{enumerate}
 \item Уточнение вида распределения каждой из описанных переменных с помощью критерия согласия 
$\chi^2$ Пирсона либо критерия согласия Колмогорова. 
 \item Исследование вида зависимости между указанными переменными (хотя бы попарно) с применением 
метода наименьших квадратов (МНК) среди нескольких производных гипотез. Например, зависимость 
линейная, квадратичная или экспоненциальная и т.д. 
 \item Выявление возможной кластеризации выборочной совокупности как для оценки возможной 
кластеризации генеральной совокупности, так и для проверки достоверности исходных данных.
\end{enumerate}

Необходимо, однако, отметить, что для каждого из этих методов выборка из 15 человек, исследованная 
в статье, может оказаться недостаточной для получения статистически достоверных результатов.  

 \printbibliography[env=gostbibliography,sorting=ntvy]
\addcontentsline{toc}{chapter}{Литература}
\end{document}

\documentclass{../../common/thesisbyxetex}

\usepackage{enumitem}
\setlist{nolistsep}	% отступы между элементами перечесления

\addbibresource{../../common/bibliography.bib}

\begin{document}

\hypersetup{
pdftitle = {Контрольная работа, история психологии},
pdfauthor = {Станкевич Наталия Александровна},
pdfsubject = {контрольная},
pdfkeywords = {Рождерс, психология, контрольная}
}% End of hypersetup

\begin{titlepage}
\newpage

\begin{center}
\large \uppercase{Белорусский государственный университет \\
факультет философии и социальных наук\\
кафедра психологии}
\end{center}
 
\vspace{12em}

\renewcommand{\mkgostheading}[1]{#1}

\begin{center}
\Large \uppercase{\textbf{Недирективная терапия К. Рождерса}}
\end{center}

\begin{center}
\textbf{Контрольная работа по дисциплине <<История психологии>>}
\end{center}

\vspace{11em}
 
\begin{flushright}
%\parbox{0.45\textwidth}{
Выполнила:

\vspace{0.25em}

студентка 1 курса отделения психологии

\textbf{Станкевич Наталия Александровна}

\vspace{2em}

Проверил:

\vspace{0.25em}

\textbf{Хонский Сергей Игоревич}

%}%end parbox
\end{flushright}

\vspace{2em}
Дата сдачи на кафедру:\underline{~~~~~~~~~~~~~~~~~}
\vspace{0.25em}

Методист:
 
\vspace{\fill}

\begin{center}
Минск, 2015
\end{center}

\end{titlepage} 

\tableofcontents 	% Оглавление, которое генерируется автоматически
\clearpage


\section{Клиент-центрированная терапия К.~Рождерса в контексте развития психологической науки в 
середине ХХ в.}
Вклад К. Рождерса в развитие психологической науки сложно переоценить. По сути, его 
клиент-центрированный (недирективный) подход послужил источником для формирования  современного 
понимания психотерапевтического процесса и развития новых методов ведения психотерапии.
Основные положения недирективного подхода в сравнении с предшествующей ему психологической 
парадигмой приведены в следующей схеме.

 \subsection*{Общая парадигма в психологии}
 
 \subsubsection*{До.} Позитивизм. Психология – наука лишь в той мере, в которой она может быть 
сведена к физиологии и, соответственно, описана с помощью естественнонаучного категориального 
аппарата, подчиняется правилам и методам естественнонаучного исследования.

\subsubsection*{После.} Объектом как теоретического изучения, так и практической работы становятся 
индивидуальные, личностные переживания конкретного человека.  Сам К.Рождерс глубоко переживал это 
противопоставление субъективного и объективного в психологии, описывая его в терминах психотерапии 
как практики и психологии как теоретической науки. <<Чем более хорошим терапевтом я становился (а я 
полагаю, так оно и было), тем более сознавал свою полную субъективность, в тех случаях, когда мне 
лучше всего удавалась эта роль. А по мере того, как я становился более опытным исследователем, 
знающим и нацеленным на решение научных проблем (а я полагаю, так оно и было), то чувствовал 
возрастающее неудобство от контраста между строгой объективностью меня-ученого и почти мистической 
субъективностью меня-терапевта>> \cite{rogersbec}. 

При этом наиболее ценные знания, - те, которые  клиент открывает о себе в ходе терапевтического 
процесса, - часто невыразимы или сложно выразимы словами, и тем более не могут быть подвергнуты 
какой бы то ни было объективации и изучению, схожему с естественнонаучным: оно не может быть 
воспроизведено или повторено в результате эесперимента, его нельзя до конца адекватно передать 
другому, наконец, ему нельзя научить, оно не может быть приобретено извне, --- а только открыто как 
результат субъективного развития.

Разрешение этого противоречия Рождерс видит в новом подходе к научному знанию: в центре как 
психотерапевтического, так и научного подходов стоит человек со всем многообразием своего 
субъективного опыта: <<существующий в данной жизни человек с его субъективностью, со всеми его 
ценностями принимается как основа и сущность и психотерапевтических, и научных отношений. И в начале 
науки также стоят человеческие отношения "Я-Ты". И в каждое из этих отношений я могу войти только 
как человек, обладающий субъективным опытом>> \cite{rogersbec}.

 \subsection*{Методы воздействия}
 
\subsubsection*{До.} В своей работе <<Консультирование и психотерапия>> К.Рождерс выделяет следующие 
методы 
психотерапевтического воздействия, относя их к устаревшим: <<приказ>> (индивиду жестко 
предписывается то или иное поведение), <<увещевание>> --- <<использование зароков и обязательств>> 
\cite[40]{rogersConsult} и <<убеждение>> --- <<Клиенту говорят: “Тебе становится лучше”, “У тебя 
улуч­шение”, “Ты хорошо себя чувствуешь”, — и все это в на­дежде усилить его мотивации в этом 
направлении>> \cite[40]{rogersConsult}. 

\subsubsection*{После.} Недирективный подход в психотерапии, собственно, не предполагает 
непосредственного воздействия на клиента. Рождерс акцентирует внимание на терапии как процессе. в 
который личностно включены как терапевт, так и пациент. Решение проблемы клиента, согласно Рождерсу, 
не принадлежит терапевту, который, зная <<как лучше>>, приказами, увещеваниями или убеждениями 
приводит к нему пациента, --- оно принадлежит именно пациенту, а задача терапевта сводится лишь к 
тому, чтобы помочь ему осознать это решение. <<Личностно-центрированный терапевт полагается на 
ресурсы клиента. Любая поза или манипуляция, например, использование эзотерического языка, 
профессионализм или диагностическое тестирование, исключаются. Это, как считаются, приводят к тому, 
что терапевт лишает клиента контроля над процессом терапии, тем самым происходит передача локуса 
оценки из рук клиента в руки терапевта и подрыв его веры в собственные способности найти пути к 
росту>> \cite{mid}.

Основным инструментом терапевтического взаимодействия становится эмпатия. которая, по словам 
Роджерса, является <<наиболее важным элементом терапии>> и представляет собой возможность <<войти в 
субъективный перцептивный мир другого и ощутить себя там, как дома. Это означает быть ежемоментно 
чутким к изменяющимся чувствам другого, к его страху, ярости, нежности, смущению и прочим 
испытываемым им чувствам. Это означает временно жить его жизнью, не делая резких движений, не 
высказывая суждений, ощущая те значения, которые сам человек почти не осознает, и не пытаясь открыть 
чувства, которые пока еще не осознаны им самим, поскольку такие попытки слишком опасны”\cite{rce}.

\subsection*{Позиция и роль терапевта в терапевтическом процессе}

\subsubsection*{До.} 

<<Терапевт знает, как решить проблему клиента. Терапевт знает как лучше>>. Б.Мидор описывает 
подобную позицию терапевта в терминах директивной психотерапии: <<Директивной терапией считается 
любая практика, в которой терапевт считается экспертом, который исходя из знания внутренних 
процессов человеческих существ ставит диагнозы и лечит тех, кто обращается к нему за помощью>> 
\cite{mid}.

\subsubsection*{После.}

<<Никто лучше клиента не знает, как решить его проблему. Терапевт лишь помогает клиенту прийти к 
решению, оптимальному для него>>. Главными требованиями к терапевтической позиции становятся 
эмпатия, конгруэнтность (подлинность) и позитивное безоценочное принятие личности клиента 
\cite{mid}. При этом если эмпатия предполагает понимание терапевтом состояния клиента, то 
конгруэнтность или подлинность --- это качество, относящееся к самому терапевту: <<Подлинность, или 
конгруэнтность - это базисная способность терапевта читать собственные внутренние переживания и 
очевидным образом проявлять их в терапевтических отношениях. Это не позволяет ему играть роль или 
демонстрировать фасад. Его слова согласуются с переживаниями. Он следует за самим собой. Он следует 
за меняющимся потоком собственных чувств и проявляет себя. В этом он прозрачен. С клиентом он в 
полной мере пытается быть самим собой>> \cite{mid}.

\subsection*{Цель терапии}

\subsubsection*{До.}

Решение конкретной проблемы, заявленной клиентом.

\subsubsection*{После.} 

Общий личностный рост, который позволит клиенту не только решить проблему, с которой он 
пришел на терапию, но и в дальнейшем решать иные проблемы, исходя из качественно нового уровня 
развития личности. Основными критериями этого роста становятся, в том числе возрастание 
независимости клиента от оценок и суждений других (перенесение локуса оценки извне в собственное 
психологическое пространство), большее внимание и доверие к себе --- своим мыслям, ощущениям,  
формирование собственной системы ценностей и более адекватное и эффективное взаимодействие с 
реальностью.

\subsection*{Основной материал для работы}

\subsubsection*{До.}

Мысли, ratio.

\subsubsection*{После.}

Чувства: <<новый терапевтический подход уделяет больше внимания эмоциональным 
факторам, чувственным аспектам ситуации, нежели интеллектуальным ее аспектам>> 
\cite[14]{rogersConsult} - 

\subsection*{Фокус терапии}

\subsubsection*{До.}
Прошлое клиента.

\subsubsection*{После.}

Его настоящее --- непосредственное переживание <<здесь и сейчас>>>: <<подобная терапия уделяет 
значительно больше внимания настоящему, а не прошлому индивида>> \cite[14]{rogersConsult}.

В этом клиент-центрированная терапия противопоставляется классическому психоанализу. Б.Мидор 
описывает это следующим образом: <<Личностно-центрированная теория фокусируется на текущем опыте 
клиента, полагая, что восстановление осознания и доверие к собственным ресурсам дает ресурсы для 
изменения и роста. В психоанализе аналитик нацелен на интерпретацию связей между прошлым и
настоящим пациента. В личностно-центрированной терапии терапевт выступает фасилитатором нахождения 
клиентом смыслов текущих внутренних переживаний>> \cite{mid}.

\subsection*{Роль терапии}

\subsubsection*{До.}

Терапия --- подготовка к дельнейшему росту клиента.

\subsubsection*{После.}

Терапевтическое взаимодействие клиента и терапевта --- это уже процесс роста.

\subsection*{Отношения к чувствам}


\subsubsection*{До.}

Бывают чувства <<хорошие>> и <<плохие>>, дозволенные и недозволенные.

\subsubsection*{После.}

Все чувства имеют права на существвание и принимаются терапевом. <<Вторая особенность 
терапевтического взаимодей­ствия — предоставление достаточной свободы выражения чувств. Вследствие 
принятия консультантом высказыва­ний клиента, полного отсутствия любых морализаторских и оценочных 
суждений, всепонимающего отношения, которое пронизывает всю беседу, клиент приходит к осоз­нанию 
того, что все его чувства и отношения могут быть выражены>> \cite[40]{rogersConsult}.

\subsection*{Роль трапевта}

\subsubsection*{До.}

Терапевт --- <<зеркало>>, не проявляет никаких эмоций (в классическом психоанализе 
вообще сидит так, чтобы пациент не видел его, поощряя тем самым развитие проекций и ассоциативного 
потока).

\subsubsection*{После.}

Основными качествами терапевта становятся эмпатия --- умение войти в эмоциональный мир клиента, 
конгруэнтность --- следование своим чувствам и внимание к ним, а также спонтанная забота о клиенте 
и его безоценочное принятие \cite{rogersConsult}.

\section{Развитие идей К.~Рождерса в современной психотерапии}

Идеи К.Рождерса, особенно его мысли о переживании клиентом своих чувств <<здесь и сейчас>>, эмпатии 
и интуиции терапевта, получили дальнейшее развитие в экзистенциальной психологии и системной 
семейной терапии. Вместе с тем, в современной терапевтической практике этиидеи получают новое 
звучание. Примером его может служить Системная семейная терапия субличностей, в настоящее время 
разрабатываемая, в том числе, американским психотерапевтом Ричардом Шварцем. В ее основу легло 
понимание К.Рождерса состояний психологической дезадаптации человека и тех механизмов, которые 
призваны ее компенсировать: <<Состояние психологической дезадаптации существует, когда организм 
либо вовсе не допускает в сознание, либо искажает какие-то значимые переживания. Вследствие этого 
точная символизация этого опыта не происходит, следовательно он не включается в гештальт 
Я-структуры, что в свою очередь приводит к неконгруэнтности между Я и опытом. Защита - это 
поведенческая реакция организма на угрозу, цель которого - сохранение нынешней структуры Я. Эта цель 
достигается путем перцептивного искажения опыта при осознании ради уменьшения неконгруэнтности между 
опытом и структурой Я или путем отрицания осознания опыта, - тем самым отрицания какой-либо угрозы 
для Я >>\cite{mid}.

В своей статье <<Depathologizing The Borderline Client>> \cite{dep} американский 
психотерапевт Ричард Шварц развивает это понимание, относя функции защиты травмированных частей 
личности различным субличностям, которые, следуя тем или иным стратегиям, стремятся предотвратить их 
ретравматизацию. Этих стратегий всего две: отстранение от окружающего мира, избегание контактов и 
отношений с другими --- и, обратно, отчаянный поиск <<спасателей>> --- людей, с которыми связывается 
надежда на излечение травмы. Для пациентов с пограничной структурой личности  эти противоположные 
тенденции развиваются одновременно с примерно равной интенсивностью, что обуславливает сложность 
выстраивания с такими пациентами терапевтического альянса. Это понимание, представляющее собой также 
попытку переосмысления психоаналитической концепции рапрошмана, позволяет Р.Шварцу по-иному 
построить терапевтический процесс, перейдя от выстраивания рамок для клиента и диалогу с каждой из 
его субличностей, переживаемому <<здесь-и-сейчас>>. Соответствующий материал сессий представлен в 
описываемой статье.

В терапии семьи акцент также делается на отношениях. Канадская исследовательница Лесли Белла ввела 
в профессиональный оборот понятие <<family making>> --- <<быть семей>>, <<строить семью>>, 
подчеркивающий значение внутрисемейных процессов и связей \cite{fam}.

\section{Выводы}

\begin{enumerate}

 \item В своем клиент-центрированном подходе Карл Рождерс кардинально переосмысляет не только 
прикладные аспекты психотерапевтической практики, но и саму парадигму, господствовавшую в 
психологической науке до его появления. Рождерс дополнил позитивистское восприятие науки 
субъективизмом терапевтической практики и доказал, что именно субъективно переживаемые клиентом 
состояния, его инсайты являются наиболее ценным материалом для работы. В связи с этим фокус терапии 
перемещается на внутренний мир клиента, формируется конепция безоценочного принятия клиента 
терапевтом, акцент делается не на осмыслении прошлого опыта, а на непосредственном переживании и 
выражении чувств здесь и сейчас.

\item Изменяются и методы психотерапевтического воздействия: на смену <<приказам>>, <<увещеваниям>> 
и <<убеждениям>> --- методам, признанным Рождерсом устаревшими, --- приходит стремление понять 
внутренний мир клиента, подчеркнуть его безусловную ценность. соответственно изменяется и роль 
терапевта: он выходит из позиции бесстрастного ученого, видящего в клиенте объект изучения, --- 
терапевтический процесс становится процессом глубинного взаимодействия между участниками.
\item Описанные изменения, на мой взгляд, существенно разнообразили методы терапевтического 
воздействия, дав развитие таким направлениям как арт-терапия, символ-драма, игровая терапия (в том 
числе с детьми), экзистенциальная психотерапия и т.д. Благодаря применению невербальных средств 
коммуникации расширилась  и аудитория потенциальных клиентов: все больший размах приобретает 
психотерапия детей, а также людей со сниженными возможностями саморефлексии и 
испытывающими сложностями в верблизации своих чувств.

В целом необходимо отметить, что идеи, выдвинутые К.Рождерсом --- как теоретические, так и 
практические --- продолжают играть важную роль в современной психологической науке. В то же время 
они подвергаются творческому пересмотру и служат основой для создания новых подходов в теории и 
практике психологии и психотерапии.
\end{enumerate}

\printbibliography[env=gostbibliography, sorting=ntvy]
\addcontentsline{toc}{chapter}{Литература}

\end{document}

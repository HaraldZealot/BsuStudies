\documentclass{../../common/thesisbyxetex}

\usepackage{enumitem}
\setlist{nolistsep}	% отступы между элементами перечесления

\addbibresource{../../common/bibliography.bib}

\begin{document}

\hypersetup{
pdftitle = {Контрольная работа, Компьютер в работе психолога},
pdfauthor = {Станкевич Наталия Александровна},
pdfsubject = {контрольная},
pdfkeywords = {психология, контрольная, компьютеры}
}% End of hypersetup

\begin{titlepage}
\newpage

\begin{center}
\large \uppercase{Белорусский государственный университет \\
факультет философии и социальных наук\\
кафедра психологии}
\end{center}
 
\vspace{14em}



\begin{center}
\Large \uppercase{\textbf{Психологические факторы формирования отцовства}}
\end{center}
\vspace{1em}

\begin{center}
\uppercase{\textbf{Курсовая работа}}
\end{center}

\vspace{10em}
 
\begin{flushright}
\parbox{0.6\textwidth}{
студентки 1 курса отделения психологии

\textbf{Станкевич Наталии Александровны}

\vspace{2em}

Научный руководитель ---

\vspace{0.25em}
кандидат психологических наук, 

доцент \textbf{Клышевич~Н.~Ю.}

}%end parbox
\end{flushright}
 
\vspace{\fill}

\begin{center}
Минск, 2015
\end{center}

\end{titlepage} 

%\tableofcontents 

%\chapter*{Введение}
%\addcontentsline{toc}{chapter}{Введение}

\chapter{Обзор литературы}

Трансформации, затронувшие институт семьи в последней трети ХХ - начале ХХI вв.,  спровоцировали 
широкую научную дискуссию, в фокусе которой оказались такие проблемы как форма семейных 
отношений \cite{gay, legfat}, фактическое содержание мужских и женских ролей \cite{mercoh, percep}, 
новые тенденции в детско-родительских отношения и отношениях в паре \cite{relot, sex}.

Одной из наиболее значимых тем в этой дискуссии становится тема родительства. Традиционно при ее 
обсуждении особое внимание уделялось матери: ее отношениям с детьми разного возраста и разного 
пола \cite{maler}, возможности совмещения материнства и профессиональной самореализации, 
социокультурным особенностям реализации материнской роли и т.д. В последнее время, однако, 
размывание гендерных ролей и серьезные изменения в организации и функционировании семьи 
вывели на первый план вопросы, связанные с ролью мужчины в этих процессах. Отцовство, будучи одной 
из ключевых сфер самоидентификации мужчины \cite{imaf}, приобретает в этой связи особую 
актуальность.

Исследования феномена отцовства могут ведутся  в различных плоскостях: можно говорить о 
психологическом, социальном, культурном измерении, о представлениях об отцовстве мужчин и ожиданиях 
и представлениях женщин о <<хорошем отце>>.

Российский исследователь И.С. Кон указывал, что отцовство может быть исследовано как <<социальный 
институт, то, как его представляет себе общество, [и] отцовство как деятельность, практики и 
стили поведения. Для исследования этих явлений нужны разные источники и методы. В первом случае 
осуществляется реконструкция и анализ социокультурных норм, чего общество ожидает от отца "вообще". 
Во втором происходит описание и анализ того, что фактически делают и чувствуют конкретные отцы, 
какова психология отцовства \cite[3]{konot}.

Созвучна этому и мысль Ю.В. Борисенко, который выделяет два подхода к изучению родительства и, в 
частности, отцовства: <<Существует два подхода к изучению родительства, в зависимости от того, кто 
считается отправной точкой изучения - ребенок или родитель. Первый, наиболее распространенный, 
подход рассматривает родительство применительно к развитию ребенка [...], во втором подходе 
рассматривается выполнение родительской роли через призму личности родителя. Здесь исследуется 
самореализация личности в родительстве, вводятся понятия «социальная роль», «статус», «социальные 
нормы», «стереотипы и требования», исследуется феномен так называемого родительского «инстинкта»
(материнского и отцовского), исследуются чувства, образы - Я, Я-концепция и другие личностные 
характеристики, так или иначе связанные и изменяющиеся с родительством \cite[11]{psyot}

В соответствии с выбранной темой курсовой работы, интерес для нас будет представлять массив 
литературы --- как теоретической, так и описывающей конкретные исследования, --- относящийся ко 
второму направлению в этой классификации, а именно к психологическому измерению отцовства. 
Социокультурные же нормы, ожидания общества мужчин в роли отцов, стереотипы и т.д. будут 
рассматриваться нами лишь в той степени, в которой они влияют на психологию отцов и конкретные 
методы реализации ими отцовских функций.

Исследования отцовства у многих авторов, как русскоязычных \cite{relot, psyot}, так и зарубежных 
\cite{meta, morfat, legfat}, начинаются с констатации факта глубинных изменений, которые происходят 
в понимании и реализации мужчинами своей роли в качестве отцов. Если традиционное понимание 
отцовской роли сводилось к отцу-<<добытчику>>, <<примеру для подражания>>, а так же 
<<персонификации власти>> в отношении детей, --- то новое видение подразумевает гораздо большую 
вовлеченность отца в процесс воспитания. В эмоциональном измерении это выражается в  эмоциональной 
ангажированности отца в процессы воспитания и взаимодействия с детьми, его близости, открытости и 
доступности для ребенка. Процессуальное измерение нового типа отцовства включат такие параметры как 
внимание, которое отец уделяет ребенку, их общие занятия, время, проведенное вместе и т.д. 
Так, исследования проведенные среди молодых матерей показали, что существенными  для них 
характеристиками <<хорошего отца>> являются: вовлеченность; помощь в уходе за ребенком; любящее и 
заботливое отношение к нему, совместные игры с ребенком; финансовая поддержка>> \cite[137]{money}.

Касательно этих тенденций исследования отцовства могут быть сгруппированы на основании той оценки, 
которую их авторы им дают: если одни воспринимают их как кризис семьи, а <<вовлеченное отцовство>> 
как вызов, то другие, как, например, И.С. Кон, предполагают, что изменение роли мужчин в семейных 
отношениях --- это часть естественного процесса размывания гендерных ролей и ослабления 
традиционного противопоставления между ними \cite{konmen}

Необходимо отметить, что подход, ориентированный на психологические факторы формирования и 
реализации отцовства, включает в себя широкий и разнообразный спектр более узких вопросов, 
исследования которых отражены в различных теоретических статьях и практических 
социально-психологических исследованиях. 

Круг психологических факторов, влияющих на то, каким 
образом человек воспринимает и реализует свои отцовские функции, весьма широк, однако 
представляется 
возможным выделить следующие наиболее существенные из них.

\section{Изменения структуры и функций института семьи}

Одним из факторов, оказывающим существенное влияние на представления об отцовстве, являются 
изменения в структуре и функциях семьи. Помимо очевидных изменений в гендерных ролях, о которых шла 
речь выше (мужчины больше вовлекаются в воспитание детей, женщины больше внимания уделяют 
работе, 
карьере и т.д.) --- меняется само видение семьи. Эти изменения можно описать следующими 
тенденциями: во-первых, меняется конфигурация семьи. На смену нуклеарной семье приходит расширенные 
ее модели --- семьи с одним родителем, бездетные семьи, семьи с однополыми родителями и т.д. 
Понимание отцовства 
в этой связи также претерпевает изменения, отходя от классической модели, в которой процедура 
установлени отцовства основывалась на предположении о том. что отцом ребенка является, скорее всего 
 тот, кто имел сексуальные отношения с его матерью, а этим человеком, вероятнее всего, 
является муж \cite[318]{legfat}

Во-вторых, происходит переход от понимания семьи как структурной единицы --- <<ячейки общества>>, 
--- к семье как процессу. На первое место выдвигаются не элементы семьи, субсистемы семейной 
системы, --- а отношения между ее членами, которые и создают то, что называется <<быть семьей>> 
\cite{fam}. В этой связи состав семьи --- возраст, пол и количество ее членов уже не так 
существенны для того, чтобы определить какую-либо группу людей как семью. Очевидно, что такое 
понимание семьи открывает новые перспективы в видении детско-родительских, в том числе 
отцовстко-детских отношений. Акцент в них делается также на отношениях, на поддержании аффективной 
связи между отцом и ребенком, что, несомненно,  вписывается в модель <<вовлеченного отцовства>>. 
Хотя вместе с тем, такое изменение отцовской роли, участие отца в родах и последующем уходе за 
ребенком представляется некоторым исследователям 
революционным \cite[15]{fatpsy}.

Вместе с тем, в центре внимания исследователей оказываются не только детско-родительские отношения. 
Проведенные исследования показывают, что для мужчин важны так же и карьерный рост, и качественный 
полноценный отдых как их индивидуальная сфера \cite{mercoh}.


\section{Семейная история}

Начальные представления человека о родительстве, в том числе и об отцовстве, начинают формироваться 
еще в детстве, в родительской семье, и напрямую связаны с качеством отношений, с одной стороны, в 
супружеской паре родителей мальчика, а с другой --- с его отношением с каждым из родителей, причем 
одни исследователи в этой связи подчеркивают роль отношений ребенка с матерью, а другие --- с 
отцом. Например, Ю.В. Борисенко, акцентируя роль матери, подчеркивает, что именно она оказывает 
существенное влияние на степень вовлеченности отца в отношения с сыном: <<... существуют 
данные, что даже в удовлетворительных брачных отношениях вовлеченность отцов, особенно когда дети 
маленькие, часто зависит от отношений и ожиданий матери, ее поддержки отцу, так же как и от степени 
ее занятости. [...]  Учитывая мощные культурные ожидания в отношении материнской роли, не 
удивительно, что активная отцовская вовлеченность в некотором смысле может угрожать женской 
идентичности>> \cite[115]{psyot}.

Другие исследования отношения к отцовству показывают важность отношений мальчика с 
отцом. Примером может служить исследование Рейчел Томпсон, показавшее,что позитивное отношение к 
собственному отцу коррелирует с желанием мужчины иметь детей и повторять модель отцовства, 
существовавшую в родительской семье: обычно те мужчины, которые позитивно высказывались о 
своих отцах, стремились повторить эту модель отцовства, а те, кто сохранил негативные 
воспоминания об отце, имели намерение исправить его ошибки, сам становясь отцом  \cite[161]{imaf}.  
Отцовство для таких мужчин играет центральную роль в жизненных 
планах и устремлениях \cite{imaf}. 

В то же время, согласно исследованиям В.К. Рахмановой, имеет место и обратная тенденция: <<Новый 
взгляд на свои отношения с отцом дает возможность молодому отцу увидеть свои отношения с отцом с 
другой стороны, заглянуть в них глубже и уже как бы изнутри, занять рефлексивную позицию по 
отношению 
к ним. Так, мужчины-отцы, по сравнению с мужчинами, не имеющими детей, чаще говорят о своих 
конфликтах с отцами, признают возможность неидеальных отношений с ними, возможность конфликтов, 
ссор, запретов и ограничений. Мужчинам, не имеющим опыта отцовства, сравнить это чувство не с чем и 
они более идеализированно и стереотипно представляют, какими они будут отцами в будущем. 
Следовательно, можно отметить, что опыт отцовства для мужчины позволяет ему во многом принять и, 
возможно, разрешить существование трудностей, сложностей в отношениях со своими отцами. Многие 
молодые отцы признают, что могут быть строгими, требовательными, могут в чем-то ошибаться, так же 
как 
ошибались их отцы \cite[54]{relot}.

\section{Личностные особенности} 

Отдельный пласт исследований посвящен специфическим случаям отцовства, когда личностные факторы 
отца могут существенно повлиять на восприятие и исполнение им отцовских функций. К таким 
исследованиям относится, например, изучение восприятия отцовства мужчинами, пережившими сексуальное 
насилие в детстве \cite{sex}, мужчин с психическими заболеваниями \cite{gbi}, алкогольной 
зависимостью и агрессивным поведением \cite{alc}, а также мужчин-геев \cite{gay}.


Характер исследований так же разнообразен, как и охватываемые ими вопросы. Проводятся как 
исследования конкретных узких вопросов, так и более масштабные лонгитюдные исследования. Примером 
может служить исследования восприятия отцами своих детей и своего нового положения в течение срока 
беременности и первых полутора лет жизни ребенка \cite{percep}. Нужно отметить, что, несмотря на 
изменяющиеся ожидания от отцовства (а возможно,именно по этой причине), --- эмоциональное состояние 
отцов в этот период скорее пониженное: эмоциональными реакциями на эту фазу часто 
становятся растерянность, смятение и удивление \cite[12]{meta}>>.


\section*{Выводы}
\addcontentsline{toc}{section}{Выводы}

\begin{enumerate}
 \item Исследования семьи, отношений в паре и детско-родительских отношений приобретают особую 
актуальность в настоящее время, когда и сама семья, и роли каждого из ее членов подвергаются 
радикальному переосмыслению.  Размытость гендерных ролей, сложные и зачастую противоречивые 
представления о должном выполнении родительских функций, заставляют во все большей мере 
фокусировать внимание на наиболее важных их аспектах.

\item Хотя, как было указано выше, в последнее время интерес исследователей к различным сторонам 
детско-родительских отношений постоянно отношений возрастает --- представляется, что большее 
значение 
придается изучению отношений в системе <<мать-дитя>>. Вопросы, связанные с 
изменениями в отцовском 
статусе изучаются не столь масштабно и комплексно. При этом, представляется, что роль отца в 
воспитании детей становится не только более разнообразной, отдаляясь от стереотипа <<кормильца --- 
примера --- олицетворения власти>>, --- но и все более значимой.

\item Современная литература, посвященная вопросам отцовства, весьма разнообразна и включает в себя 
как теоретические работы, так и прикладные исследования. Изучаются вопросы, связанные с 
самовосприятием мужчин в роли отцов, ожидания от отцов со стороны общества, различные аспекты, 
связанные с прикладным выполнением отцовских функций.

\item Одним из направлений исследования отцовства является изучение тех психологических факторов, 
которые оказывают влияние на формирование представлений об отцовстве и реализацию отцовских функций 
у мужчин. Среди таких факторов представляется возможным выделить следующие:
\begin{itemize}
 \item семейная история (точнее, субъективное восприятие ее мужчиной)
 \item личностные особенности мужчины (включая тип личности, жизненный опыт, ценностные ориентации 
и т.д.)
 \item Конфигурацию и отношения в созданной им семье.
\end{itemize}

\end{enumerate}

%\chapter{Изменение структуры и функций семьи последней трети XX начале XXI века}

%\chapter{Роль семейной истории в восприятии отцовства и реализации отцовских функций}

%\chapter{Личностные характеристики мужчины, как фактор формирования отцовства}

%\chapter*{Заключение}
%\addcontentsline{toc}{chapter}{Заключение}

\printbibliography[title={Список использованных источников},env=gostbibliography,sorting=ntvy]
\addcontentsline{toc}{chapter}{Список использованных источников}


\end{document}

